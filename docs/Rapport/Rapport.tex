\input{../tex/preambule}

\title{\vspace{\fill}\textbf{\Huge Rapport}}
\author{
	Sonny Klotz - Idir Hamad - Younes Benyamna - Malek Zemni
	\vspace{2em}\\
	\textit{Projet M1 Informatique}\\\textit{Primalité}
	\vspace{2em}
}

\begin{document}
\pagenumbering{gobble}\clearpage
\maketitle\vspace{9em}
\begin{center}\includegraphics[scale=0.7]{logo.png}\end{center}
\begin{flushright}Module \textit{TER}\end{flushright}
\newpage
\tableofcontents
\newpage\clearpage\pagenumbering{arabic}

	\section*{Introduction}
	
	\paragraph{}Ce document est le compte-rendu final de notre travail qui s'inscrit dans le cadre du module \textit{Projet} du master informatique de l'\textit{\textbf{UVSQ}}.
	
	
	\section{Architecture de l'application}
		\subsection{Organigramme et données échangées}
		Cet organigramme préalablement établi, représente la décomposition en modules de l'application, ainsi que les informations qui circulent entre ces modules.
		
		\subsection{Fonctionnalités des modules}
		
	\section{Outils et langages de programmation}
		\subsection{Contraintes}
		Lors de la réalisation de l'application, des contraintes ont dû être respectées.
		Ces contraintes se sont avérées déterminantes pour les choix techniques que l'on a fait, et en l'occurrence, le choix des langages de programmation. 
		
		\subsection{Choix des outils et des langages}
		Les contraintes imposées et les exigences définies nous ont permis de fixer nos choix sur plusieurs langages de programmation qui vont interagir ensemble :
	
	\section{Fonctionnement de l'application}
	
	\section{Cryptosystèmes}
	
	\section{Tests de primalité}

		
	\section{Bilan technique du projet}
		Notre produit final, c'est à dire l'application, se comporte comme prévu : l'application est fonctionnelle, la liaison entre ses différents modules réussit bien et les différentes fonctionnalités fournissent le résultat attendu.		

		\subsection{Problèmes rencontrés}
			\subsubsection*{Problèmes résolus :} 
			Lors de la réalisation de l'application, on a été confrontés à plusieurs problèmes et points délicats, principalement des verrous techniques, qui ont perturbé le bon déroulement de notre travail :
				
			\subsubsection*{Problèmes non résolus :}
				Certains problèmes rencontrés n'ont pas été entièrement résolus. Ces problèmes ne sont pas déterminants pour l'acceptabilité de notre produit.
				
	\section{Organisation interne du groupe}
	Assignation des modules pour chaque membre du groupe :
	Cette répartition a été parfaitement respectée. Elle nous a permis de travailler efficacement et assez indépendamment, ce qui prouve que l'assignation des modules a été judicieusement faite. Nous sommes également restés en contact pendant toute la phase de développement pour s'entraider pour la prise en main des nouveaux outils.
	
	\section{Coûts}
	Ce tableau indique les coûts estimés et les coûts finaux, en nombre de lignes de code et pour chaque module :
	
	\section*{Conclusion}
		
\end{document}
