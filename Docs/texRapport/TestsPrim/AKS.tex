\subsection{Test AKS}
		

	\subsubsection{Algorithme}
		\paragraph{}
		\vspace{-1.5em}\begin{adjustwidth}{1.5cm}{1.5cm} 
		\begin{Def}[Ordre multiplicatif]
			\label{Ordre}
			En théorie des nombres , donné un entier a et un entier positif n avec pgcd(a,n) = 1, l'ordre multiplicatif d'un modulo n est le plus petit entier positif k avec $a^{k} \equiv 1$ $(mod$ $n)$
		\end{Def}
		\end{adjustwidth}\vspace{0.5em}
		
		\paragraph{}
		\vspace{-1.5em}\begin{adjustwidth}{1.5cm}{1.5cm} 
		\begin{algorithm}[H]
			\caption{Test AKS}\label{TAKS}
			\Donnees{un entier $n > 1$ }
			
			1 - \Si {$n = a^{b}$ pour $a \in N$ et $b > 1$}{\Retour composé\;}
			
            2 - Déterminer le plus petit entier r tel que l’ordre de n dans $\mathbb{Z}/r\mathbb{Z}$ soit supérieur à $log(n)^{2}$\;
            3 - \Si {$1 < pgcd(a,n) < n$ pour un entier $a \leq r$} {\Retour composé\;}
            
            4 - \Si {$n \leq r$}{\Retour premier\;}
            
            5 - \Pour{$a = 1$ jusqu'à $\lfloor\sqrt{\varphi(r)} log(n)\rfloor{}{}$}{
                \Si {$(X + a)^{n} \not\equiv X^{n} + a$ dans $ \frac{\mathbb{Z}/n\mathbb{Z}[X]}{ (X^{r} - 1)\mathbb{Z}/n\mathbb{Z}[X]}$}{
					\Retour composé\;
                }
            }
            
			6 - \Retour premier\;
		\end{algorithm}
		\end{adjustwidth}\vspace{0.5em}
		
	\subsubsection{Principe de la preuve}
		\paragraph{}Le test AKS se base sur une version généralisée du petit théorème de Fermat :
		\vspace{-1.5em}\begin{adjustwidth}{1.5cm}{1.5cm} 
		\begin{Th}[Petit théorème de Fermat généralisé]
			\label{ThFermat3}
			Soient $a \in \mathbb{Z}$ et $n \in \mathbb{N}^{*}$ tels que $pgcd(a,n)=1$, alors
			\[n \text{ premier} \iff (X + a)^{n} \equiv X^{n} + a \text{ mod } n\] 
		\end{Th}
		\end{adjustwidth}\vspace{0.5em}
		
		\paragraph{}Ce théorème nous donne un test déterministe très simple mais il faut évaluer les $n$ coefficients du polynôme $(X + a)^{n}$ à l'aide du binôme de Newton, ce qui est bien trop long.
		\paragraph{}La solution proposée dans le test est de réduire le polynôme modulo $X^{r} - 1$ avec $r$ bien choisi. Cependant avec cette modification, même si l'implication du théorème "$\Rightarrow$" est vérifiée, ce n'est plus le cas pour la réciproque.
		\paragraph{}Ce problème est géré dans les premières étapes du test. En effet, si l'équation est vérifée pour $r$ bien choisi et un nombre suffisant de $a$ (obtenus en temps polynomiaux) alors $n$ est une puissance de nombre premier, c'est-à-dire, $n = p^{b}$ avec $p$ premier et $b \in \mathbb{N}^{*}$.
		\paragraph{}On obtient finalement un test déterministe polynomial en la taille de l'entrée $n$.
		
	\subsubsection{Complexité}
		\paragraph{}En supposant que les additions, multiplications et divisions s'effectuent toutes en $log(n)$, l'ordre de grandeur de la complexité temporelle prouvée par les auteurs du test est $O(log(n)^{15/2})$.
		\paragraph{}Une borne moins précise mais démontrable plus facilement est $O(log(n)^{21/2})$ c'est la première complexité prouvée et elle a donc servi à établir que trouver un nombre premier est un problème que l'on peut résoudre en temps polynomial.
		\paragraph{}La complexité repose sur l'étape 5 de l'algorithme. En effet r est borné par $log(n)^{5}$, cela dit, sous certaines conjectures non prouvées (Artin et Sophie-Germain) cette borne est réduite à $log(n)^2$ ce qui améliore la complexité de l'algorithme en $O(log(n)^6)$. On retient cette complexité car les preuves des conjectures sont sur la bonne voie.
		\paragraph{}Finalement, si la conjecture d'Agrawal est vérifée, ce test peut être amélioré en $O(log(n)^{3})$ ce qui le rendrait comparable aux tests probabilistes utilisés en cryptographie. 
