\subsection{Test de Wilson}
	Le test de primalité de Wilson est un test déterministe basé sur le théorème suivant :
		
		\vspace{-1.5em}\begin{adjustwidth}{1.5cm}{1.5cm} 
		\begin{Th}[Théorème de Wilson]
			un entier $n > 1$ est un nombre premier si et seulement si
			\[(n-1)! + 1 \equiv 0 \pmod n\]
		\end{Th}
		\end{adjustwidth}\vspace{0.5em}
		
	Ce théorème fournit une caractérisation des nombres premiers assez anecdotique et ne constitue pas un test de primalité efficace. Son principal intérêt réside dans son histoire et dans la relative simplicité de son énoncé et de ses preuves.\\
	En effet, ce théorème était connu à partir du XVII\up{e} siècle en Europe. En 1770, \textit{John Wilson} redécouvre une conjecture de ce théorème et la publie. Ensuite, les mathématiciens \textit{Lagrange}, \textit{Euler} et \textit{Gauss} le démontrent chacun à son tour. 
		
	\subsubsection*{Algorithme}
		L'algorithme du test de primalité basé sur le théorème de Wilson est le suivant :\\
		
		\begin{algorithm}[H]
			\caption{Test de Wilson}\label{Wil}
			\Donnees{un entier $n$}
			\Si{$(n-1)! + 1 \equiv 0 \pmod n$} { 
				{\Retour premier\;}
			} \Sinon{
				{\Retour composé\;}
			}
		\end{algorithm}	
		
	\subsubsection*{Complexité}
		Ce test qui est basé sur une propriété très simple a cependant une complexité trop élevée. Il faut effectuer environ $n$ multiplications modulaires, par conséquent la complexité est de $O(n)$.
