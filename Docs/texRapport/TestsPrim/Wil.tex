\subsection{Test de Wilson}
	Le \textit{\textbf{test de Wilson}} est basé sur le théorème suivant :
		
		\vspace{-1.5em}\begin{adjustwidth}{1.5cm}{1.5cm} 
		\begin{Th}[Théorème de Wilson]
			un entier $n > 1$ est un nombre premier si et seulement si
			\[(n-1)! \equiv -1 \pmod n\]
		\end{Th}
		\end{adjustwidth}\vspace{0.5em}
		
		
		
		\subsubsection*{Complexité}
			Ce test qui est basé sur une propriété très simple a cependant une complexité trop élevée. Il faut effectuer environ $n$ multiplications modulaires, par conséquent la complexité est de $O(n)$.
