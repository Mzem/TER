\section{Mesures de performance et comparatifs}
	Dans la partie précédente, on s'est contenté de présenter différents tests de primalité sans conclure par rapport à leurs performances. Dans cette partie, on va mesurer ces performances et établir un comparatif entre les différents tests abordés.
	
	\subsection{Évolution des tests de primalité}
		\colorbox{yellow}{Sonny développe cette partie quand tu fais les complexités}\\
		Plusieurs algorithmes de tests de primalité se sont succédés au fil du temps. Des algorithmes de plus en plus efficaces apparaissent pour en remplacer d'autres.
	
		\subsubsection*{Premiers tests déterministes}
			Les premiers algorithmes de tests de primalité apparus sont les algorithmes de \textit{\textbf{test naïf}}. Ces algorithmes sont déterministes, c'est-à-dire qu'ils retournent toujours une réponse exacte. Ils constituent la façon la plus naturelle de tester la primalité d'un nombre. Cependant, leur complexité était trop élevée, surtout quand il s'agissait des tester de grands nombres.
			
			\paragraph{} D'autres algorithmes déterministes très simples et un peu plus performant sont aussi apparus, telle que le \textit{\textbf{test de Wilson}}. Mais la complexité est restée trop élevée surtout pour de grands nombres.
		
		\subsubsection*{Tests probabilistes}
			Les algorithmes de tests probabilistes constituent une façon beaucoup plus efficaces que les premiers algorithmes déterministes découverts pour tester la primalité d'un nombre. Ces algorithmes de type \textit{Monte-Carlo} décident si un entier est premier ou pas avec une certaine probabilité $P$. La sortie d'un test de primalité probabiliste sur un entier $n$ est soit :
			\begin{itemize}
				\item $n$ est composé : toujours vrai
				\item $n$ est premier : vrai avec une probabilité $P$
			\end{itemize}
			Dans le cas où la sortie du test est "premier", il y a toujours une petite probabilité que le nombre testé ne soit pas vraiment premier. Pour pallier à ce problème et diminuer cette probabilité, on a tendance à répéter le test probabiliste un certain nombre de fois.
			
			\paragraph{}Le premier algorithme probabiliste apparu est le \textit{\textbf{test de Fermat}}. Ce test qui repose sur un théorème énoncé en 1640 a longement était utilisé en pratique, jusqu'à l'apparition du \textit{\textbf{test de Solovay-Strassen}} en 1977. 
			
			\paragraph{} À partir de 1980, le \textit{\textbf{test de Solovay-Strassen}} a été lui aussi remplacé en pratique par le \textit{\textbf{test de Miller-Rabin}}, plus efficace, car reposant sur un critère analogue, mais ne donnant de faux positif qu'au plus une fois sur quatre lorsque le nombre testé n'est pas premier.
			
		\subsubsection*{Tests déterministes rapides : AKS}
		
	
	\subsection{Mesure du temps d'exécution}
		
		
		
		
	\subsection{Test de primalité optimal}
		
