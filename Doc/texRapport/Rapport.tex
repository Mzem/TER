\input{../texLib/preambule}
\usepackage[french,frenchkw,ruled,vlined]{../texLib/algorithm2e}

\title{\vspace{\fill}\textbf{\Huge Rapport}}
\author{
	Sonny Klotz - Idir Hamad - Younes Benyamna - Malek Zemni
	\vspace{2em}\\
	\textit{Projet M1 Informatique}\\\textit{Primalité}
	\vspace{2em}
}

\declaretheorem[name=Théorème]{Th}
\declaretheorem[name=Définition]{Def}


\begin{document}
\pagenumbering{gobble}\clearpage
\maketitle\vspace{9em}
\begin{center}\includegraphics[scale=0.7]{img/logo.png}\end{center}
\begin{flushright}Module \textit{TER}\end{flushright}

\newpage
\tableofcontents

\newpage
\listoffigures
\listofalgorithms
\renewcommand{\listtheoremname}{Liste des théorèmes et des définitions}
\listoftheorems

\newpage\clearpage\pagenumbering{arabic}



	\section*{Introduction}
	
	\paragraph{}Ce document est le compte-rendu final de notre projet sur les tests de primalité qui s'inscrit dans le cadre du module \textit{TER} du M1 informatique de l'\textit{UVSQ}.
	\paragraph{}Les tests de primalité sont des algorithmes qui permettent de savoir si un nombre entier est premier. Ces tests sont indispensables pour la cryptographie à clé publique.
	\paragraph{}Il existe plusieurs algorithmes de tests de primalité, plus ou moins performants. L'efficacité de ces algorithmes est particulièrement liée à la taille des données entrées. 
	\paragraph{}Notre travail consiste donc à implémenter différents tests de primalité et de comparer leurs performances. Ces mesures de performance nous permettront de construire un algorithme qui génère un nombre premier de manière optimale. Cet algorithme sera le produit final de l'application développée dans ce projet. 
	\paragraph{}Dans la première partie de ce document, on présentera l'architecture de notre application, illustrée par un organigramme.
	\paragraph{}Dans la deuxième partie, on évoquera les champs d'application principaux des tests de primalité, en l'occurrence, les cryptosystèmes à clé publique.
	\paragraph{}Dans la troisième partie, on détaillera le processus utilisé en pratique pour générer un nombre premier.
	\paragraph{}Ensuite, la quatrième partie traitera des tests de primalité qu'on implémentera : leur algorithme, leur complexité et leur preuve.
	\paragraph{}Finalement, on établira dans la dernière partie un comparatif entre ces algorithmes de tests de primalité ainsi qu'un processus de mesure de performance qui aboutira à une génération optimale d'un nombre premier.
	
	
	\section{Architecture de l'application}

	\subsection{Organigramme et données échangées}
	L'application développée dans ce projet porte essentiellement sur l'implémentation des tests de primalité, avec quelques fonctionnalités supplémentaires. La manière dont l'application est structurée va permettre de faciliter son extension et sa réutilisation.\\
	\indent Cet organigramme représente la décomposition en modules de l'application ainsi que les informations qui circulent entre ces modules.
	\begin{figure}[H]
		\begin{tikzpicture}
		\begin{scope}[xscale=2,yscale=0.9]
			
			\node (Fct) at (0,5) [rectangle,draw,text depth=2cm,minimum width=12cm,minimum height=2.5cm,font=\textbf\Large] {\begin{tabular}{c}Fonctionnalités\end{tabular}};
			\node (CS) [rectangle,draw,dashed] at ([xshift=-2cm]Fct.center) {\begin{tabular}{c}Cryptosystèmes\end{tabular}};
			\node (TPrim) [rectangle,draw,dashed] at ([xshift=2cm]Fct.center) {\begin{tabular}{c}Tests de primalité\end{tabular}};
		
			\node (MPerf) at (0,1) [rectangle,draw,text depth=-1.5cm,minimum width=5cm,minimum height=2.5cm,font=\textbf\Large] {\begin{tabular}{c}Mesures de performance\end{tabular}};
			\node (MTps) [rectangle,draw,dashed] at ([yshift=0.3cm]MPerf.center) {\begin{tabular}{c}Mesure temps\end{tabular}};
			
			\draw[-triangle 45,blue!60] (CS.2.5) -- node[anchor=south,yshift=0.0cm]{1} (TPrim.177.5);
			\draw[-triangle 45,blue!60] (TPrim.182.5) -- node[anchor=south,yshift=-0.5cm]{2} (CS.357.5);
			
			\draw[-triangle 45,blue!60] (MTps.60) -- node[anchor=north,yshift=0.0cm]{3} (TPrim.300);
			\draw[-triangle 45,blue!60] (TPrim.250) -- node[anchor=south,xshift=-0.5cm,yshift=-0.3cm]{4} (MTps.110);
		\end{scope}
		%Légende
		\begin{scope}
			\node (LEGENDE) at (-7,-2) {\textbf{Légende :}};
			\node (PACKAGE) at (-4.5,-2) [rectangle,draw] {\begin{tabular}{c}Package\end{tabular}};
			\node (MODULE) at (-2,-2) [rectangle,draw,dashed] {\begin{tabular}{c}Module\end{tabular}};
			\path[->,>=stealth',blue!60] (0.5,-2.3) edge[bend left=0] node[anchor=south,above]{informations transmises} (3,-2.3);
		\end{scope}
		\end{tikzpicture}
		\caption{Organigramme des différents modules de l'application}\label{fig:M1}
	\end{figure}
		
	\vspace{1em}
	\hspace{-1.3em}\textbf{Notes :}\\
		\textbf{(1)} Nombre entier à tester (primalité)\\
		\textbf{(2)} Réponse sur la primalité (0 composé, 1 probablement premier, 2 premier)\\
		\textbf{(3)} Nombre entier à tester (primalité)\\
		\textbf{(4)} Réponse sur la primalité (0 composé, 1 probablement premier, 2 premier)\\
		
	
	\subsection{Fonctionnalités des modules}
		\subsubsection*{Package Fonctionnalités}
			\begin{enumerate}[leftmargin=*]
				\item Module Cryptosystèmes : implémentation de cryptosystèmes ayant recours à des nombres premiers (\textbf{\textit{RSA}}) et des générateurs aléatoires de nombres premiers (RPNG) :
				\begin{itemize}
					\item Cryptosystème RSA
					\item RPNG avec test de primalité déterministe
					\item RPNG avec test de primalité probabiliste
					\item RPNG optimal
				\end{itemize}
				\item Module Tests de primalité : implémentation de différents algorithmes de test de primalité qui feront l'objet d'une étude comparative par la suite :
				\begin{itemize}
					\item Test Naïf
					\item Test de Wilson
					\item Test de Fermat
					\item Test de Miller-Rabin
					\item Test de Solovay-Strassen
					\item Test AKS
				\end{itemize}
			\end{enumerate}
			
		\subsubsection*{Package Mesures de performance}
			\begin{enumerate}[leftmargin=*]
				\item Module Mesure temps : mesure du temps d'exécution de chaque test de primalité selon le la taille en bits du nombre à tester.
			\end{enumerate}
	
	\subsection{Outils et langages de programmation}
	Notre application va être implémentée dans le langage {\ttfamily C}. Le langage {\ttfamily C} possède plusieurs types pour représenter des nombre entiers. Cependant, tous ces types ont une précision fixe et ne peuvent pas dépasser un certain nombre d'octets. Le type le plus grand est le {\ttfamily long long int} qui peut contenir des entiers d'un taille maximale de 64 bits. Or, tous ces types sont beaucoup trop courts pour les applications cryptographiques qui nécessitent la manipulation de données d'au moins 512 bits.
	\paragraph{}{\ttfamily GNU MP} pour {\ttfamily GNU Multi Precision}, souvent appelée \lstinline!GMP! est une bibliothèque {\ttfamily C}/{\ttfamily C++} de calcul multiprécision sur des nombres entiers, rationnels et à virgule flottante qui permet en particulier de manipuler de très grand nombres.
	
	\paragraph{} Finalement, le logiciel {\ttfamily Gnuplot} va être utilisé pour faire des représentations graphiques à partir des résultats issus des mesures de performance de notre application.
	
	
	\section{Cryptosystèmes - RSA}

	Les tests de primalité sont des algorithmes indispensables pour la cryptographie à clé publique. Ces tests sont couramment utilisés par les cryptosystèmes \textbf{\textit{RSA}} et \textbf{\textit{ElGamal}} afin de générer des nombres premiers.\\
	Pour \textit{RSA}, les tests sont effectués lors la phase de génération de clés. Pour \textit{ElGamal}, ils sont effectués lors de l'établissement d'un échange de clés.\\
	Dans cette partie, on va détailler le cryptosystème \textit{RSA} et exhiber rôle important des nombres premiers. On a choisi de s'intéresser qu'à un seul cryptosystème puisque le choix du cryptosystème n'aura pas d'effet sur les performances des tests de primalité, objet principal de ce projet.
	
	\subsection{Description de RSA}
	Décrit en 1977 par Ronald Rivest, Adi Shamir et Leonard Adleman, RSA est un cryptosystème basé sur le problème de factorisation, qui utilise une paire de clés (publique, privée) permettant de chiffrer et de déchiffrer un message. Le fonctionnement de RSA peut être décrit en 3 phases :
		\begin{enumerate}[leftmargin=2em]
			\vspace{1em}
			\item \textbf{Génération des clés} 
			\begin{itemize}
				\item Choisir 2 grands \textbf{\textit{nombres premiers}} distincts $p$ et $q$.
				\item Calculer $n = p * q$. $n$ est le module RSA et fait 1024 bits au minimum en général.
				\item Calculer $\Phi(n) = (p - 1)(q - 1)$.
				\item Choisir $e \in \mathbb{Z}_{\Phi(n)}^{*}$ ($e$ premier avec $\Phi(n)$).
				\item Calculer $d$ telle que $d*e \equiv 1 mod \Phi(n)$ ($d$ inverse de $e$ pour la multiplication modulo $\Phi(n)$).
			\end{itemize}
			Les éléments échangés constituant la clé publique sont $(n, e)$. Les éléments constituant la clé privé sont $(p, q, d)$.
			\vspace{1em}
			\item \textbf{Chiffrement}\\
			Pour chiffrer un message $M$ en un chiffré $C$, on utilise les éléments de la clé publique $(n, e)$ :
			\[C \equiv M^{e} \pmod n\]		
			
			\item \textbf{Déchiffrement}\\
			Pour déchiffrer un chiffré $C$ en un message clair $M$, on utilise les éléments de la clé privée $(p, q, d)$ :
			\[M \equiv C^{d} \pmod n\]
		\end{enumerate}
		
		\subsection{Rôle des nombres premiers}
		La première étape pour la mise en place d'un cryptosystème RSA est la génération de deux très grands nombres premiers $p$ et $q$. Leur produit $n = p * q$ forme le module RSA. Pour cette raison, la taille de $p$ et $q$ en bits, doit être égale à la moitie de la taille en bits du module $n$. Par exemple, dans le cadre de RSA-1024, les deux nombres premiers doivent avoir une longueur de 512 bits.
		\paragraph{}En effet, un attaquant qui connait le module RSA $n$ et la clé publique $e$ doit connaitre la factorisation de $n$ en nombres premiers pour trouver la clé privée $d$. Ainsi, l'entier $n$ doit être très grand afin que sa factorisation ne soit pas possible avec les ressources de calcul actuelles. On voit donc l'intérêt crucial pour la sécurité de générer les deux grands nombres premiers $p$ et $q$.
		\paragraph{}Parmi les algorithmes classiques de factorisation les plus efficaces, on retrouve \textbf{\textit{GNFS}} (General Number Field Sieve) dont le temps d'exécution croît exponentiellement à la taille de $n$ (complexité exponentielle). Avec les puissances de calcul actuelles, il est de plus en plus déconseillé d'utiliser un module RSA de taille 1024 bits. Il est estimé qu'un module de taille 2048 bits soit sécurisé (complexité factorisation supérieure à $2^{80}$) jusqu'à l'année 2020. 
		En 1994, l'algorithme de Shor appliqué sur des ordinateurs quantiques a permis d'effectuer un factorisation en un temps non exponentiel. Les applications des ordinateurs quantiques permettent théoriquement de casser RSA par la force brute, mais actuellement ces ordinateurs génèrent des erreurs aléatoires qui les rendent inefficaces.
		
	
	\section{Génération des nombres premiers}

	Les cryptosystèmes utilisent une approche commune pour la génération des nombres premiers. Cette approche générale consiste à utiliser un générateur de nombres aléatoires pour générer un entier, dont on testera ensuite la primalité. Ce processus est illustré par la figure ci-dessous :
	
	\begin{figure}[H]
		\begin{center}
		\begin{tikzpicture}
		\begin{scope}
			\node (Gen) at (0,0) [rectangle,draw] {\begin{tabular}{c}Générateur de\\nombres aléatoires\end{tabular}};	
			\node (Test) at (6,0) [rectangle,draw] {\begin{tabular}{c}Test de\\primalité\end{tabular}};	
			\node (Prim) at (10,1) [rectangle] {\begin{tabular}{c}$\bar{p}$ est premier\end{tabular}};	
			\node (Comp) at (10,-1) [rectangle] {\begin{tabular}{c}$\bar{p}$ est composé\end{tabular}};	
			
			\draw[-triangle 45] (Gen) -- node[anchor=south]{Candidat $\bar{p}$} (Test);
			\draw[-triangle 45] (Test) -- node[anchor=south]{} (Prim.180);
			\draw[-triangle 45] (Test) -- node[anchor=south]{} (Comp.180);
		
		\end{scope}
		\end{tikzpicture}
		\end{center}
		\caption{Processus de génération des nombres premiers}\label{fig:M2}
	\end{figure}
	Dans cette démarche, il est important d'utiliser un bon générateur de nombres aléatoires, qui ne doit dans aucun cas être prévisible. Si un attaquant réussit à deviner les nombres premiers qui composent le module RSA, alors le système est immédiatement cassé.

	\subsubsection*{Fréquence des nombres premiers}
	Lors de la génération des nombres premiers à l'aide de ce processus, on voudrait savoir combien de nombres doit-on tester avant de trouver un nombre premier. La réponse à cette question est donnée par le \textit{Théorème des Nombres Premiers}.
	
		\vspace{-1.5em}\begin{adjustwidth}{1.5cm}{1.5cm} 
		\begin{Th}[Théorème des Nombres Premiers]
			Soit $\pi(n)$ le nombre de premiers qui sont inférieurs à $n$, alors
			\[\pi(n) \approx \frac{n}{ln(n)} \quad \quad (n \to +\infty)\]
		\end{Th}
		\end{adjustwidth}\vspace{0.5em}
		
	Un graphique de la fonction $\pi(n)$ pour les 1000 premiers nombres premiers est donné dans la figure ci-dessous :
	\begin{figure}[H]
		\begin{center}\includegraphics[scale=0.4]{img/freqPremiers.png}\end{center}\vspace{-3em}
		\caption{La fonction $\pi(n)$ pour les 1000 premiers nombres premiers}\label{fig:M3}
	\end{figure}
	
	Le tableau suivant contient l'approximation ainsi que le nombre exact de nombres premiers pour différentes valeurs de $n$. On y remarque que l'approximation est assez bonne.
	\begin{table}[H]\begin{center}
		\begin{tabular}{|lll|}
		\hline
		$n$  & $n/ln(n)$ & $\pi(n)$     \\ \hline
		$10^{3}$ & $145$     & $168$     \\
		$10^{4}$ & $1 086$   & $1 229$   \\
		$10^{5}$ & $8 686$   & $9 592$   \\
		$10^{6}$ & $72 382$  & $78 498$  \\
		$10^{7}$ & $620 420$ & $664 579$ \\ \hline
		\end{tabular}
	\end{center}\end{table}
	
	\paragraph{Probabilité de générer un nombre premier $p$ de $k$ bits :} on sait que $2^{k-1} \leqslant p \leqslant 2^{k}-1$. Le nombre de nombres premiers dans cet intervalle (c-à-d de $k$ bits) peut être approximé par :
	\[ \pi(2^{k}) - \pi(2^{k-1}) \approx \frac{2^{k}}{ln(2^{k})} - \frac{2^{k-1}}{ln(2^{k-1})} \approx \frac{2^{k-1}}{ln(2^{k-1})}	\]
	puisque $ln(2^{k}) = ln(2.2^{k-1}) = ln(2) + ln(2^{k-1})$, donc $ln(2^{k}) \approx ln(2^{k-1})$ pour $k$ grand.
	\vspace{1em}
	\\
	\noindent Donc, il y'a $2^{k-1}$ entiers $\in [2^{k-1},2^{k}-1]$ (de $k$ bits) dont approximativement $\frac{2^{k-1}}{ln(2^{k-1})}$ parmi eux qui sont premiers. Par conséquent, un nombre $p$ de $k$ bits sera premier avec une probabilité de :
	\[\frac{1}{ln(2^{k-1})}\]
	
	\paragraph{Cas de RSA-1024 :} pour générer une des deux clé de RSA-1024 dont la taille en bits est 512, on a probabilité de $1/ln(2^{511}) \approx 1/355$ pour générer un nombre premier de 512 bits. Cette chance double si on se restreint sur les entiers impairs, c'est à dire qu'on doit générer à peu près 177 nombres avant de tomber sur un nombre premier.
	
	\section{Tests de primalité}

	Les tests de primalité interviennent dans la deuxième étape du processus de génération des nombres premiers. Ce sont des algorithmes qui permettent de savoir si un nombre entier est premier. Dans le cas où le nombre n'est pas premier, il est dit \textbf{\textit{composé}}. Dans cette partie, on va détailler différents algorithmes de tests de primalité.\\
	Les tests de primalité peuvent être :
	\begin{itemize}[leftmargin=*]
		\item \textbf{déterministes :} fournissent toujours la même réponse pour un nombre donné
		\item \textbf{probabilistes :} peuvent fournir des réponses différentes pour un même nombre (utilisent des données tirées aléatoirement)
	\end{itemize}
	
	Voici la liste des différents algorithmes de tests de primalité qu'on va plus ou moins aborder :
	\begin{table}[H]\begin{center}
		\begin{tabular}{|c|c|c|}
		\hline
		Algorithme           & Année & Type       \\ \hline
		Naïf (Crible d’Eratosthène) & -240  & Déterministe \\ \hline
		Fermat               & 1640  & Probabiliste \\ \hline
		Wilson               & 1770  & Déterministe \\ \hline
		Miller-Rabin         & 1976  & Probabiliste \\ \hline
		Solovay-Strassen     & 1977  & Probabiliste \\ \hline
		AKS                  & 2002  & Déterministe \\ \hline
		\end{tabular}
	\end{center}\end{table}
	
	Certains tests seront énoncés rapidement du fait qu'il ne sont pas assez performants. Par contre, on s'intéressera plus en détail aux tests de \textit{Fermat}, \textit{Miller-Rabin}, \textit{Solovay-Strassen} et \textit{AKS}. Pour chacun de ces tests, on donnera un bref historique, son algorithme, sa complexité, sa preuve, ainsi que son implémentation.
	
		\subsection{Test naïf - Crible d’Eratosthène}
		Le test naïf représente l'idée la plus intuitive pour tester la primalité d'un nombre entier. Pour décider si un nombre $n$ est premier ou composé, on teste si les entiers $2, 3, ..., n-1$ divisent $n$. Si un parmi ces entiers divise $n$ alors on déduit que $n$ est composé, sinon on conclut qu'il est premier. Ceci revient à factoriser le nombre en question.\\
		Pour améliorer cet algorithme, on sait qu'un diviseur d'un entier $n$ quelconque ne peut dépasser $n/2$. De plus, si $n$ possède un diviseur plus grand que $\sqrt{n}$, alors il a forcement au moins un diviseur plus petit que $\sqrt{n}$. On peut donc accélérer la recherche en ne prenant en compte que des nombres premiers inférieurs à $\sqrt{n}$. Pour cela il suffit de pré-calculer et de stoker dans une table tous les nombres premiers $\leqslant \sqrt{n}$. Le \textbf{\textit{crible d'Eratosthène}} par exemple peut être utilisé dans ce but.\\
	
		\begin{algorithm}[H]
			\caption{Test naïf}\label{TN}
			\Donnees{un entier $n$}
			\Pour{tout nombre premier $p \leqslant \sqrt{n}$}{
				\Si {$p$ divise $n$}
					{\Retour composé\;}
			}
		\Retour premier\;
		\end{algorithm}
		
		\subsubsection*{Crible d'Eratosthène}
		Ce crible est un procédé établi par Eratosthène, un mathématicien grec  du III\up{e} siècle av. J.-C., qui permet de trouver tous les nombres premiers inférieurs à un certain entier naturel donné $N$. Dans notre cas, cet entier donné est $\sqrt{n}$, $n$ étant le nombre dont on va tester la primalité.\\
		L'algorithme procède par élimination : il s'agit de supprimer d'une table des entiers de $2$ à $N$ tous les multiples d'un entier. En supprimant tous les multiples, à la fin il ne restera que les entiers qui ne sont multiples d'aucun entier, et qui sont donc les nombres premiers.
		\begin{itemize}
			\item retirer les multiples du plus petit entier premier restant (multiples de 2, puis de 3, etc.)
			\item on peut s'arrêter lorsque le carré de ce plus petit entier premier restant est supérieur au plus grand entier premier restant, car dans ce cas, tous les non-premiers ont déjà été retirés précédemment
			\item à la fin du processus, tous les entiers qui n'ont pas été rayés sont les nombres premiers inférieurs à $N$
		\end{itemize}
		L'algorithme du crible est le suivant :\\ 
		
		\begin{algorithm}[H]
			\caption{Crible d'Eratosthène}\label{Eras}
			\Donnees{un entier $N$ qui correspond à $\sqrt{n}$}
			{Créer une liste L de couples \textit{(entier, primalité)}, pour les entiers allant de $2$ jusqu'à $N$, avec une primalité initialisée à "premier" : \textit{\textbf{L = \{(2, premier), (3, premier), ..., ($N$, premier)\}}} \;}
			{\textit{\textbf{plusGrandPremier}} = $N$ \;}
			\Pour{tout nombre $p$ marqué "premier" de la liste L (de manière croissante)}{
				\Si{$p^{2}$ > \textit{\textbf{plusGrandPremier}}}{ 
					\Retour L\;
				}
				{\textit{\textbf{i}} = $2$ \;}
				\Tq{$p*i < N$}{ 
					{Marquer "composé" l'entier à la position $p*i$ \;}
					{Mettre à jour \textit{\textbf{plusGrandPremier}} \;}
					{$i++$ \;}
				}
			}
		\end{algorithm}	
			
			
		\subsubsection*{Complexité}
			La complexité en temps de l'algorithme \ref{TN} (Test naïf) dans le pire des cas est de $\pi(n) \approx \frac{2\sqrt{n}}{ln(n)}$ division, c'est-à-dire $O(\sqrt{n})$ opérations. Dans le cas de RSA-1024, la complexité de cette méthode avoisine les $2^{503}$ divisions.
			
	\subsection{Test de Wilson}
	Le \textit{\textbf{test de Wilson}} est basé sur le théorème suivant :
		
		\vspace{-1.5em}\begin{adjustwidth}{1.5cm}{1.5cm} 
		\begin{Th}[Théorème de Wilson]
			un entier $n > 1$ est un nombre premier si et seulement si
			\[(n-1)! \equiv -1 \pmod n\]
		\end{Th}
		\end{adjustwidth}\vspace{0.5em}
		
		
		
		\subsubsection*{Complexité}
			Ce test qui est basé sur une propriété très simple a cependant une complexité trop élevée. Il faut effectuer environ $n$ multiplications modulaires, par conséquent la complexité est de $O(n)$.

	\subsection{Test de Fermat}
	Le test de Fermat est un test de primalité probabiliste basé sur le \textit{petit théorème de Fermat} :
	
	\vspace{-1.5em}\begin{adjustwidth}{1.5cm}{1.5cm} 
	\begin{Th}[Petit théorème de Fermat (énoncé 1)]
		\label{ThFermat1}
		Si $p$ est un nombre premier, alors pour tout nombre entier $a$ premier avec $p$
		\[a^{p-1}\equiv 1 \pmod p\]
	\end{Th}
	\end{adjustwidth}\vspace{0.5em}
	
	Il existe un énoncé équivalent de ce théorème, qui est le suivant :
	
	\vspace{-1.5em}\begin{adjustwidth}{1.5cm}{1.5cm} 
	\begin{Th}[Petit théorème de Fermat (énoncé 2)]
		\label{ThFermat2}
		Si $p$ est un nombre premier, et $a$ un nombre entier quelconque, alors
		\[a^{p}\equiv a \pmod p\]
	\end{Th}
	\end{adjustwidth}\vspace{0.5em}
	
	Ce théorème doit son nom à \textit{Pierre de Fermat}, qui l'énonce la première fois en 1640. 
	
	\subsubsection{Algorithme}
		Le premier énoncé du théorème de Fermat va être exploité pour construire l'algorithme du test de primalité. Ce théorème décrit une propriété commune à tous les nombres premiers qui peut être utilisée pour détecter si un nombre est premier ou bien composé.\\
		En effet, pour un entier $n$ dont on veut tester la primalité et un entier $a$ quelconque telle que $1 < a < n$ : 
		\begin{itemize}
			\item Le fait de choisir $1 < a < n$ garantit que si $n$ était premier, $a$ sera forcément premier avec $n$ (puisque $a < n$) et ainsi le test n'échouera pas. 
			\item Si $\mathbf{a^{n-1} \not\equiv 1 \pmod n}$, alors $n$ est surement composé.\\
			Parmi les entiers $a$ qui ne vérifient pas l'inégalité de Fermat, il y a évidement ceux qui ne sont pas premiers avec $n$. Si l'on trouve un tel entier $a$ (qu'il soit premier ou non avec $n$), on dit que $a$ est un \textit{\textbf{témoin de non primalité}} de $n$ issu de la divisibilité (\textit{témoin de Fermat)}.
					
					\vspace{-1.5em}\begin{adjustwidth}{1.5cm}{1.5cm} 
					\begin{Def}[Témoin de Fermat]
						\label{TemFermat}
						Soit un entier $n \geqslant 2$. On appelle témoin de Fermat pour $n$, tout entier $a$, telle que
						\[1 < a < n  \quad \text{et} \quad a^{n-1} \not\equiv 1 \pmod n\]
					\end{Def}
					\end{adjustwidth}\vspace{0.5em}
					
			\item Si $\mathbf{a^{n-1}\equiv 1 \pmod n}$, on ne peut pas conclure avec certitude que $n$ est premier puisque la réciproque du \textit{théorème de Fermat est fausse} (théorème \ref{ThFermat1}).\\
				Un nombre $n$ vérifiant cette équation peut être premier, mais aussi composé, dans ce cas $n$ est dit \textit{\textbf{pseudo-premier} de base $a$} ou menteur.
					
					\vspace{-1.5em}\begin{adjustwidth}{1.5cm}{1.5cm} 
					\begin{Def}[Nombre pseudo-premier]
						\label{PseudoPrem}
						Un nombre pseudo-premier est un nombre premier probable (un entier naturel qui partage une propriété commune à tous les nombres premiers) qui n'est en fait pas premier. Un nombre pseudo-premier provenant du théorème de Fermat est appelé nombre pseudo-premier de Fermat.
					\end{Def}
					\end{adjustwidth}\vspace{0.5em}
					
				Si un nombre pseudo-premier $n$ de base $a$ est pseudo-premier pour toutes les valeurs de $a$ qui sont premières avec $n$ est appelé \textit{\textbf{nombre de Carmichael}}. 
			
					\vspace{-1.5em}\begin{adjustwidth}{1.5cm}{1.5cm} 
					\begin{Def}[Nombre de Carmichael]
						\label{Carmich}
						Un entier positif composé $n$ est appelé nombre de Carmichael si pour tout entier $a$ premier avec $n$,
						\[a^{n-1}\equiv 1 \pmod n\]
					\end{Def}
					\end{adjustwidth}\vspace{0.5em}
					
				L'entier $n = 561 = 3\ .\ 11\ .\ 17$ est le plus petit nombre de Carmichael puisque $a^{560} \equiv 1 \pmod 561$ pour tout entier $a$ premier avec $561$. Les nombres de Carmichael sont très rares. Il existe par exemple seulement $246\ 683$ nombres de Carmichael inférieurs à $10^{16}$. Le nombre de premiers inférieurs à $10^{16}$ est quant à lui égal à $279\ 238\ 341\ 033\ 925$. Donc la probabilité qu'un nombre premier inférieur à $10^{16}$ soit un nombre de Carmichael est plus petite que $1/10^{9}$.
			
		\end{itemize}
		
		\paragraph{}Les nombres pseudo-premiers et les nombre de Carmichael sont relativement rares. On peut donc envisager d'adopter ce critère pour un test probabiliste de primalité, qui est le test de Fermat. En effet, va être être répété $k$ fois, et à chaque itération, on effectue le test avec une base $a$ différente. Plus le nombre de répétitions est grand, plus la probabilité que le résultat du test soit correct augmente.\\
		
		\begin{algorithm}[H]
			\caption{Test de Fermat}\label{TF}
			\Donnees{un entier $n$ et le nombre de répétitions $k$}
			\Pour{$i$ = $1$ jusqu'à $k$}{
				Choisir aléatoirement $a$ tel que $1 < a < n$\;
				\Si {$a^{p-1} \not\equiv 1 \pmod n$}
					{\Retour composé\;}
			}
		\Retour probablement premier\;
		\end{algorithm}
		
		
	\subsubsection{Complexité}
		Si un algorithme rapide est utilisé pour l'exponentiation modulaire (par exemple Square\&Multiply), la complexité en temps de l'algorithme de Fermat est
		\[O(k\ .\ log_{2}(n)\ .\ log(log(n))\ .\ log(log(log(n))))\]
		où $k$ est le nombre de fois que le test est répété.
	
	
	\subsubsection{Preuve}
		\paragraph{}Pour démontrer cet algorithme nous allons d'abord prouver le petit théorème de Fermat. Pour cela nous nous basons sur le fait que c'est un cas particulier du théorème d'Euler :
		
		\vspace{-1.5em}\begin{adjustwidth}{1.5cm}{1.5cm} 
		\begin{Th}[Théorème d'Euler]
			\label{ThEuler}
			Soit $n$ un naturel supérieur ou égal à 2, et $a$ un entier premier avec $n$, alors
			\[a^{\phi(n)}\equiv 1 \pmod n\]
			où $\phi$ est la fonction indicatrice d'Euler :
			\begin{align*}
				\phi \colon \mathbb{N} &\to \mathbb{N}\\
				n &\mapsto | \{k \text{, } 1 \leq k \leq n \text{ et } pgcd(k,n) = 1\} |
			\end{align*}
		\end{Th}
		\end{adjustwidth}\vspace{0.5em}
		
		La fonction $\phi$ évaluée sur un nombre premier $p$ vaut $p - 1$. Le petit théorème de Fermat est donc une application du théorème d'Euler en remplaçant $n$ par un nombre premier $p$.
		
		\paragraph{Preuve du théorème d'Euler :\\}
			En prouvant le théorème d'Euler, nous aurons donc prouvé petit le théorème de Fermat, nous conclurons ensuite avec le preuve de l'algorithme.
			\paragraph{}Nous effectuerons nos calculs dans le groupe multiplicatif $(\mathbb{Z}/n\mathbb{Z})^*$, l'ensemble des naturels inférieurs à $n$ inversibles modulo $n$, ou de manière équivalente, l'ensemble des naturels inférieurs à $n$ premiers avec $n$.
			
			\paragraph{}Considérons l'application suivante, avec $\alpha \in (\mathbb{Z}/n\mathbb{Z})^*$:
				\begin{align*}
					\Gamma_{\alpha} \colon (\mathbb{Z}/n\mathbb{Z})^* &\to (\mathbb{Z}/n\mathbb{Z})^*\\
					x &\mapsto \alpha \cdot x
				\end{align*}
				C'est une bijection. En effet son application inverse est $\Gamma_{\alpha^{-1}}$. $\alpha \in (\mathbb{Z}/n\mathbb{Z})^*$, donc il existe $\alpha^{-1} \in (\mathbb{Z}/n\mathbb{Z})^*$ inverse de $\alpha$ modulo $n$. C'est également une permutation (bijection d'un ensemble vers lui-même), on a donc :
				\begin{align*}
					\prod_{x \in (\mathbb{Z}/n\mathbb{Z})^*} x &= \prod_{x \in (\mathbb{Z}/n\mathbb{Z})^*} \Gamma_{\alpha}(x)\\
					&= \alpha^{|(\mathbb{Z}/n\mathbb{Z})^* |} \cdot \prod_{x \in (\mathbb{Z}/n\mathbb{Z})^*} x\\
					&= \alpha^{\phi(n)} \cdot \prod_{x \in (\mathbb{Z}/n\mathbb{Z})^*} x
				\end{align*}
				$\prod_{x \in (\mathbb{Z}/n\mathbb{Z})^*} x$ est inversible (produit d'éléments inversibles), donc on simplifie :
				\[\alpha^{\phi(n)} \equiv 1 \text{ modulo } n \]
		
		\paragraph{Preuve de l'algorithme :\\}
		
		\paragraph{Remarques :\\}
		

	\subsection{Test de Miller-Rabin}
	Le test de Miller-Rabin est un autre test de primalité probabiliste basé sur le \textit{petit théorème de Fermat} (théorème \ref{ThFermat}). Il exploite quant à lui quelques propriétés supplémentaires.

	\subsection{Test de Solovay-Strassen}

	Le test de Solovay-Strassen est un test de primalité probabiliste, publié par \textit{Robert Solovay} et \textit{Volker Strassen} en 1977. Ce test a une importance historique dans la démonstration de la faisabilité du cryptosystème RSA.

	\subsubsection{Algorithme}
		L'algorithme du test de Solovay-Strassen est essentiellement basé sur le \textit{\textbf{critère d'Euler}}, un théorème qui énonce que :
		\vspace{-1.5em}\begin{adjustwidth}{1.5cm}{1.5cm} 
		\begin{Th}[Critère d'Euler]
			\label{CritereEuler}
			Soient $p > 2$ un nombre premier et $a$ un entier premier avec $p$
			\begin{itemize}
				\item Si $a$ est un résidu quadratique modulo $p$, alors $a^{\frac{n-1}{2}} \equiv 1 \pmod p$.
				\item Si $a$ n'est pas un résidu quadratique modulo $p$, alors $a^{\frac{n-1}{2}} \equiv -1 \pmod p$.
			\end{itemize}
			Ceci se résume en utilisant le symbole de Legendre par :
			\[a^{\frac{n-1}{2}} \equiv \left ( \frac{a}{p} \right ) \pmod p\]
		\end{Th}
		\end{adjustwidth}\vspace{0.5em}
		
		\vspace{-1.5em}\begin{adjustwidth}{1.5cm}{1.5cm} 
		\begin{Def}[Résidu quadratique]
			\label{Residu}
			On dit qu'un entier $q$ est un résidu quadratique modulo $p$ s'il existe un entier $x$ tel que :
			\[x^{2} \equiv q \pmod p\]
			Autrement dit, un résidu quadratique modulo $p$ est un nombre qui possède une racine carrée de module $p$. Dans le cas contraire, on dit que $q$ est un non-résidu quadratique modulo $p$.
		\end{Def}
		\end{adjustwidth}\vspace{0.5em}
		
		Le \textit{\textbf{symbole de Legendre}} est utilisé pour résumer le \textit{critère d'Euler}. Il est définit de la manière suivante :
		\vspace{-1.5em}\begin{adjustwidth}{1.5cm}{1.5cm} 
		\begin{Def}[Symbole de Legendre]
			\label{Legendre}
			Le symbole de Legendre est une fonction de deux variables entières à valeurs dans $\{-1, 0, 1\}$. Si $p$ est un nombre premier et $a$ un entier, alors le symbole de Legendre $\left ( \frac{a}{p} \right )$ vaut :
			\[
			\left\{
			\begin{array}{l l}
			0 & \text{si } a \text{ est divisible par } p\\
			1 & \text{si } a \text{ est un résidu quadratique modulo } p \text{ mais pas divisible par } p\\
			-1 & \text{si } a \text{ n'est pas un résidu quadratique modulo } p
			\end{array}
			\right.
			\]
			Le cas particulier $p = 2$ est inclus dans cette définition mais est sans intérêt : $\left ( \frac{a}{p} \right )$ vaut $0$ si $a$ pair et $1$ sinon.
		\end{Def}
		\end{adjustwidth}\vspace{0.5em}
		
		Pour pouvoir exploiter le \textit{critère d'Euler} dans l'algorithme du test de primalité, on doit pouvoir calculer le \textit{symbole de Legendre} pour tout entier $n$ dont on veut tester la primalité. On introduit donc le \textit{\textbf{symbole de Jacobi}} qui est une généralisation du \textit{symbole de Legendre}, définit de la manière suivante :
		\vspace{-1.5em}\begin{adjustwidth}{1.5cm}{1.5cm} 
		\begin{Def}[Symbole de Jacobi]
			\label{Jacobi}
			Le symbole de Jacobi $\left ( \frac{a}{n} \right )$ est défini, $\forall a \in \mathbb{Z}$ et $n \in \mathbb{N}$ \underline{impair}, comme produit de symboles de Legendre, en faisant intervenir la décomposition en facteurs premiers de $n$. Si $n = p_{1}*p_{2}*...*p_{k}$ pour $k \in \mathbb{N}$ telle que $p_{1}, p_{2}, ...,p_{k}$ sont des nombres premiers non nécessairement distincts, alors :
			\[\left ( \frac{a}{n} \right ) = \left ( \frac{a}{\prod_{i \in \{1,...,k\}} p_{i}} \right ) = \prod_{i \in \{1,...,k\}} \left ( \frac{a}{p_{i}} \right )\]
		\end{Def}
		\end{adjustwidth}\vspace{0.5em}
		
		Compte tenu des notions décrites ci-dessus, on peut maintenant construire l'algorithme de test de primalité de Solovay-Strassen :\\
		
		\begin{algorithm}[H]
			\caption{Test de Solovay-Strassen}\label{TSS}
			\Donnees{un entier $n$ \underline{impair} et le nombre de répétitions $k$}
			\Pour{$i$ = $1$ jusqu'à $k$}{
				Choisir aléatoirement $a$ tel que $2 < a < n$\;
				$x \gets \left ( \frac{a}{n} \right )$\;
				\Si {$x = 0$ ou $x \not\equiv a^{\frac{n-1}{2}} \pmod n$}{
					{\Retour composé\;}
				}
			}
		\Retour probablement premier\;
		\end{algorithm}
		
		\paragraph{} Cet algorithme exploite essentiellement le \textit{critère d'Euler} (théorème \ref{CritereEuler}). En effet, pour un entier $n$ dont on veut tester la primalité et un entier $a$ quelconque telle que $2 < a < n$ :
		\begin{itemize}
		\item Si $\mathbf{a^{\frac{n-1}{2}} \not\equiv \left ( \frac{a}{n} \right ) \pmod n}$, alors $n$ est surement composé.\\
		Parmi les entiers $a$ qui ne vérifient pas le \textit{critère d'Euler} (théorème \ref{CritereEuler}), il y a évidement ceux qui ne sont pas premiers avec $n$. Si l'on trouve un tel entier $a$ (qu'il soit premier ou non avec $n$), on dit que $a$ est un \textit{\textbf{témoin de non primalité}} de $n$ (\textit{témoin d'Euler}).
			
			\vspace{-1.5em}\begin{adjustwidth}{1.5cm}{1.5cm} 
			\begin{Def}[Témoin d'Euler]
			\label{TemEuler}
				Soit un entier $n > 2$. On appelle témoin d'Euler pour $n$, tout entier $a$, telle que
				\[2 < a < n  \quad \text{et} \quad a^{\frac{n-1}{2}} \not\equiv \left ( \frac{a}{n} \right ) \pmod n\]
			\end{Def}
			\end{adjustwidth}\vspace{0.5em}
		
		\item Si $\mathbf{a^{\frac{n-1}{2}} \equiv \left ( \frac{a}{n} \right ) \pmod n}$, on ne peut pas conclure avec certitude que $n$ est premier puisque la réciproque du \textit{critère d'Euler} (théorème \ref{CritereEuler}) est fausse.\\
		Un nombre $n$ vérifiant cette équation peut être premier, mais aussi composé, dans ce cas $n$ est dit \textit{\textbf{pseudo-premier d'Euler-Jacobi} de base $a$} ou menteur.
					
			\vspace{-1.5em}\begin{adjustwidth}{1.5cm}{1.5cm} 
			\begin{Def}[Nombre pseudo-premier d'Euler-Jacobi]
				\label{PseudoPremEulerJ}
				Un nombre pseudo-premier d'Euler-Jacobi de base $a$ est un nombre composé impair $n$ premier avec $a$ et tel que la congruence suivante soit vérifiée :
				\[a^{\frac{n-1}{2}} \equiv \left ( \frac{a}{n} \right ) \pmod n\]
			\end{Def}
			\end{adjustwidth}\vspace{0.5em}
			
		À la différence du test de primalité de Fermat, pour chaque entier composé $n$, au moins la moitié de tous les $a$ sont des témoins d’Euler. Par conséquent, il n’y a aucune valeur de $n$ pour laquelle tous les $a$ sont des menteurs, alors que c'est le cas pour les nombres de \textit{Carmichael} dans le test de Fermat.
		\end{itemize}
	
	
	\subsubsection{Complexité}
		\paragraph{}Pour étudier la complexité du test Solovay-Strassen, il faut étudier l'évaluation du symbole de Jacobi. En effet, dans le corps de la boucle, à la différence du test de Fermat, avant d'effectuer l'exponentiation modulaire nous allons évaluer un symbole de Jacobi. La complexité d'une itération sera de l'ordre du terme dominant entre le symbole de Jacobi et l'exponentiation modulaire.
		\paragraph{}Le symbole de Jacobi $\left ( \frac{a}{b} \right )$ s'évalue en $O(log(a) \cdot log(b))$. Ainsi, dans le cadre de notre test, le symbole de Jacobi s'évalue en $O(log(n)^{2})$.
		\paragraph{}L'évaluation du symbole de Jacobi reste dominée par l'exponentiation rapide même couplée d'une multiplication modulaire rapide.\\
				On rappelle de complexités en $O(log(n)^{2} \cdot log(log(n)) \cdot log(log(log(n))))$ en utilisant la multiplication FFT et en $ O(log(n)^{1 + log_{2}(3)})$ (avec $1 + log_{2}(3) > 2$) avec la muliplication de Karatsuba.
				
	\subsubsection{Preuve}
		\paragraph{}Durant cette démonstration, nous allons utiliser et admettre le théorème de Lagrange.
		\vspace{-1.5em}\begin{adjustwidth}{1.5cm}{1.5cm} 
		\begin{Th}[Théorème de Lagrange]
			\label{ThLagrange}
			Soient $p$ un nombre premier et $f(X) \in \mathbb{Z}[X]$ un polynôme à coefficients entiers alors:
			\begin{itemize}
			\item Soit tous les coefficients de $f$ sont divisibles par $p$.
			\item Soit $f(X) \equiv 0 \text{ mod } p$ admet au plus $degre(f)$ solutions non équivalentes.
			\end{itemize}
		\end{Th}
		\end{adjustwidth}\vspace{0.5em}
		
		\paragraph{}En admettant le théorème de Lagrange, il suffit de prouver le critère d'Euler pour avoir une preuve satisfaisante du théorème. En effet, comme énoncé précédemment, le test de Solovay-Strassen s'appuie directement sur la contraposée du critère d'Euler.
		\paragraph{}Nous partirons de plusieurs constatations pour prouver le critère d'Euler :\\
		\begin{enumerate}
		\item D'après le théorème de Lagrange, comme $p$ est premier, $x^{2} \equiv a \text{ mod } p$ admet au plus deux solutions distinctes pour chaque $a$ différent. Donc, hormis $x = 0$, comme chaque racine $x$ peut être accompagnée d'une deuxième racine comme solution de l'équation, il y a au moins $(p - 1)/2$ résidus quadratiques $a$ différents. 
		\item $(\mathbb{Z}/p\mathbb{Z}, +, \cdot)$ est un corps ($p$ premier).
		\end{enumerate}
		
		\paragraph{}Pour commencer on va partir du théorème de Fermat et le réécrire.
		\[a^{p-1} \equiv 1 \text{ mod } p \iff (a^{\frac{p-1}{2}} - 1) \cdot (a^{\frac{p-1}{2}} + 1) \equiv 0 \text{ mod } p\]
		Grâce à la remarque (2), on obtient que le produit est nul si et seulement si l'un au moins des facteurs est nul.
		
		\paragraph{}Si $a$ est un résidu quadratique, il existe $x$ tel que $x^{2} \equiv a \text{ mod } p$, on a :\\
		\[
			a^{\frac{p-1}{2}} \equiv (x^{2})^{\frac{p-1}{2}} \equiv x^{p-1} \equiv 1 \text{ mod } p
		\]
		La dernière étape est obtenue à l'aide du petit théorème de Fermat.
		
		\paragraph{}On sait que d'après le théorème de Lagrange, $(a^{\frac{p-1}{2}} - 1) \equiv 0 \text{ mod } p$ admet au plus $(p-1)/2$ solutions pour $a$. On sait également (constatation 1) qu'il y a au moins $(p-1)/2$ résidus quadratiques modulo $p$.
		\paragraph{}Donc, il y a exactement $(p-1)/2$ valeurs qui annulent le premier facteur : les résidus quadratiques. Et, les autres $(p-1)/2$ valeurs non-résidues annulent forcément le second terme pour satisfaire le petit théorème de Fermat.
		\paragraph{}En résumé :
			\begin{itemize}
				\item Si $a$ est un résidu quadratique modulo $p$, alors $a^{\frac{n-1}{2}} \equiv 1 \pmod p$.
				\item Si $a$ n'est pas un résidu quadratique modulo $p$, alors $a^{\frac{n-1}{2}} \equiv -1 \pmod p$.
			\end{itemize}

	\subsection{Test AKS}
		

	\subsubsection{Algorithme}
		\colorbox{yellow}{Younes + Idir + Sonny}
		
	\subsubsection{Complexité}
		\colorbox{yellow}{Sonny}
	
	\subsubsection{Preuve}
		\colorbox{yellow}{Sonny}
	
	\section{Mesures de performance et comparatifs}
	Dans la partie précédente, on s'est contenté de présenter différents tests de primalité sans conclure par rapport à leurs performances. Dans cette partie, on va mesurer ces performances et établir un comparatif entre les différents tests abordés.
	
	\subsection{Évolution des tests de primalité}
		\colorbox{yellow}{Sonny développe cette partie quand tu fais les complexités}\\
		Plusieurs algorithmes de tests de primalité se sont succédés au fil du temps. Des algorithmes de plus en plus efficaces apparaissent pour en remplacer d'autres.
	
		\subsubsection*{Premiers tests déterministes}
			Les premiers algorithmes de tests de primalité apparus sont les algorithmes de \textit{\textbf{test naïf}}. Ces algorithmes sont déterministes, c'est-à-dire qu'ils retournent toujours une réponse exacte. Ils constituent la façon la plus naturelle de tester la primalité d'un nombre. Cependant, leur complexité était trop élevée, surtout quand il s'agissait des tester de grands nombres.
			
			\paragraph{} D'autres algorithmes déterministes très simples et un peu plus performant sont aussi apparus, telle que le \textit{\textbf{test de Wilson}}. Mais la complexité est restée trop élevée surtout pour de grands nombres.
		
		\subsubsection*{Tests probabilistes}
			Les algorithmes de tests probabilistes constituent une façon beaucoup plus efficaces que les premiers algorithmes déterministes découverts pour tester la primalité d'un nombre. Ces algorithmes de type \textit{Monte-Carlo} décident si un entier est premier ou pas avec une certaine probabilité $P$. La sortie d'un test de primalité probabiliste sur un entier $n$ est soit :
			\begin{itemize}
				\item $n$ est composé : toujours vrai
				\item $n$ est premier : vrai avec une probabilité $P$
			\end{itemize}
			Dans le cas où la sortie du test est "premier", il y a toujours une petite probabilité que le nombre testé ne soit pas vraiment premier. Pour pallier à ce problème et diminuer cette probabilité, on a tendance à répéter le test probabiliste un certain nombre de fois.
			
			\paragraph{}Le premier algorithme probabiliste apparu est le \textit{\textbf{test de Fermat}}. Ce test qui repose sur un théorème énoncé en 1640 a longement était utilisé en pratique, jusqu'à l'apparition du \textit{\textbf{test de Solovay-Strassen}} en 1977. 
			
			\paragraph{} À partir de 1980, le \textit{\textbf{test de Solovay-Strassen}} a été lui aussi remplacé en pratique par le \textit{\textbf{test de Miller-Rabin}}, plus efficace, car reposant sur un critère analogue, mais ne donnant de faux positif qu'au plus une fois sur quatre lorsque le nombre testé n'est pas premier.
			
		\subsubsection*{Tests déterministes rapides : AKS}
		
	
	\subsection{Mesure du temps d'exécution}
		
		
		
		
	\subsection{Test de primalité optimal}
		

	
	
	\section*{Conclusion}
		Ce document est le rapport final de notre projet sur les tests de primalité. Ce travail a abouti à une implémentation de divers tests de primalité, une description de leur principe et de leur preuve, et une comparaison des performances de ces tests.
		\paragraph{} On peut revoir, avec du recul, la manière avec laquelle notre travail a été effectué. En effet, les majeures difficultés ont été principalement trouvées lors de l'établissement des complexité et surtout des preuve des algorithmes implémentés. En ce qui concerne l'implémentation, le \textit{test AKS} a été le plus difficile, on a donc eu à réutiliser du code libre pour construire l'algorithme final.
		\paragraph{} Certains points restent encore ouverts. On pourrait envisager d'appliquer d'autres méthodes de mesure de performance pour établir non pas un générateur optimal, mais un test de primalité optimal. Il faudrait aussi effectuer des mesures sur les tests probabilistes pour déterminer le nombre optimal de répétitions à effectuer. Cela reviendrait à voir plus en détail les probabilités d'erreur de ces tests.
		\paragraph{} Ce projet a eu la particularité de faire appel à des compétences en informatique et en mathématique de manière égale. Cette épreuve constitue donc une expérience enrichissante pour nous, sur le plan des compétences, mais aussi sur le plan du travail collectif, qui nous inspirera certainement pour de futurs projets.
	
		
\end{document}
