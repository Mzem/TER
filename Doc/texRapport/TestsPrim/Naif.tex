	\subsection{Test naïf - Crible d’Eratosthène}
		Le test naïf représente l'idée la plus intuitive pour tester la primalité d'un nombre entier. Pour décider si un nombre $n$ est premier ou composé, on teste si les entiers $2, 3, ..., n-1$ divisent $n$. Si un parmi ces entiers divise $n$ alors on déduit que $n$ est composé, sinon on conclut qu'il est premier. Ceci revient à factoriser le nombre en question.\\
		Pour améliorer cet algorithme, on sait qu'un diviseur d'un entier $n$ quelconque ne peut dépasser $n/2$. De plus, si $n$ possède un diviseur plus grand que $\sqrt{n}$, alors il a forcement au moins un diviseur plus petit que $\sqrt{n}$. On peut donc accélérer la recherche en ne prenant en compte que des nombres premiers inférieurs à $\sqrt{n}$. Pour cela il suffit de pré-calculer et de stoker dans une table tous les nombres premiers $\leqslant \sqrt{n}$. Le \textbf{\textit{crible d'Eratosthène}} par exemple peut être utilisé dans ce but.\\
	
		\begin{algorithm}[H]
			\caption{Test naïf}\label{TN}
			\Donnees{un entier $n$}
			\Pour{tout nombre premier $p \leqslant \sqrt{n}$}{
				\Si {$p$ divise $n$}
					{\Retour composé\;}
			}
		\Retour premier\;
		\end{algorithm}
		
		\subsubsection*{Crible d'Eratosthène}
		Ce crible est un procédé établi par Eratosthène, un mathématicien grec  du III\up{e} siècle av. J.-C., qui permet de trouver tous les nombres premiers inférieurs à un certain entier naturel donné $N$. Dans notre cas, cet entier donné est $\sqrt{n}$, $n$ étant le nombre dont on va tester la primalité.\\
		L'algorithme procède par élimination : il s'agit de supprimer d'une table des entiers de $2$ à $N$ tous les multiples d'un entier. En supprimant tous les multiples, à la fin il ne restera que les entiers qui ne sont multiples d'aucun entier, et qui sont donc les nombres premiers.
		\begin{itemize}
			\item retirer les multiples du plus petit entier premier restant (multiples de 2, puis de 3, etc.)
			\item on peut s'arrêter lorsque le carré de ce plus petit entier premier restant est supérieur au plus grand entier premier restant, car dans ce cas, tous les non-premiers ont déjà été retirés précédemment
			\item à la fin du processus, tous les entiers qui n'ont pas été rayés sont les nombres premiers inférieurs à $N$
		\end{itemize}
		L'algorithme du crible est le suivant :\\ 
		
		\begin{algorithm}[H]
			\caption{Crible d'Eratosthène}\label{Eras}
			\Donnees{un entier $N$ qui correspond à $\sqrt{n}$}
			{Créer une liste L de couples \textit{(entier, primalité)}, pour les entiers allant de $2$ jusqu'à $N$, avec une primalité initialisée à "premier" : \textit{\textbf{L = \{(2, premier), (3, premier), ..., ($N$, premier)\}}} \;}
			{\textit{\textbf{plusGrandPremier}} = $N$ \;}
			\Pour{tout nombre $p$ marqué "premier" de la liste L (de manière croissante)}{
				\Si{$p^{2}$ > \textit{\textbf{plusGrandPremier}}}{ 
					\Retour L\;
				}
				{\textit{\textbf{i}} = $2$ \;}
				\Tq{$p*i < N$}{ 
					{Marquer "composé" l'entier à la position $p*i$ \;}
					{Mettre à jour \textit{\textbf{plusGrandPremier}} \;}
					{$i++$ \;}
				}
			}
		\end{algorithm}	
			
			
		\subsubsection*{Complexité}
			La complexité en temps de l'algorithme \ref{TN} (Test naïf) dans le pire des cas est de $\pi(n) \approx \frac{2\sqrt{n}}{ln(n)}$ division, c'est-à-dire $O(\sqrt{n})$ opérations. Dans le cas de RSA-1024, la complexité de cette méthode avoisine les $2^{503}$ divisions.
			