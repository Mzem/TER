\subsection{Test de Solovay-Strassen}

	Le test de Solovay-Strassen est un test de primalité probabiliste, publié par \textit{Robert Solovay} et \textit{Volker Strassen} en 1977. Ce test a une importance historique dans la démonstration de la faisabilité du cryptosystème RSA.

	\subsubsection{Algorithme}
		L'algorithme du test de Solovay-Strassen est essentiellement basé sur le \textit{\textbf{critère d'Euler}}, un théorème qui énonce que :
		\vspace{-1.5em}\begin{adjustwidth}{1.5cm}{1.5cm} 
		\begin{Th}[Critère d'Euler]
			\label{CritereEuler}
			Soient $p > 2$ un nombre premier et $a$ un entier premier avec $p$
			\begin{itemize}
				\item Si $a$ est un résidu quadratique modulo $p$, alors $a^{\frac{n-1}{2}} \equiv 1 \pmod p$.
				\item Si $a$ n'est pas un résidu quadratique modulo $p$, alors $a^{\frac{n-1}{2}} \equiv -1 \pmod p$.
			\end{itemize}
			Ceci se résume en utilisant le symbole de Legendre par :
			\[a^{\frac{n-1}{2}} \equiv \left ( \frac{a}{p} \right ) \pmod p\]
		\end{Th}
		\end{adjustwidth}\vspace{0.5em}
		
		\vspace{-1.5em}\begin{adjustwidth}{1.5cm}{1.5cm} 
		\begin{Def}[Résidu quadratique]
			\label{Residu}
			On dit qu'un entier $q$ est un résidu quadratique modulo $p$ s'il existe un entier $x$ tel que :
			\[x^{2} \equiv q \pmod p\]
			Autrement dit, un résidu quadratique modulo $p$ est un nombre qui possède une racine carrée de module $p$. Dans le cas contraire, on dit que $q$ est un non-résidu quadratique modulo $p$.
		\end{Def}
		\end{adjustwidth}\vspace{0.5em}
		
		Le \textit{\textbf{symbole de Legendre}} est utilisé pour résumer le \textit{critère d'Euler}. Il est définit de la manière suivante :
		\vspace{-1.5em}\begin{adjustwidth}{1.5cm}{1.5cm} 
		\begin{Def}[Symbole de Legendre]
			\label{Legendre}
			Le symbole de Legendre est une fonction de deux variables entières à valeurs dans $\{-1, 0, 1\}$. Si $p$ est un nombre premier et $a$ un entier, alors le symbole de Legendre $\left ( \frac{a}{p} \right )$ vaut :
			\[
			\left\{
			\begin{array}{l l}
			0 & \text{si } a \text{ est divisible par } p\\
			1 & \text{si } a \text{ est un résidu quadratique modulo } p \text{ mais pas divisible par } p\\
			-1 & \text{si } a \text{ n'est pas un résidu quadratique modulo } p
			\end{array}
			\right.
			\]
			Le cas particulier $p = 2$ est inclus dans cette définition mais est sans intérêt : $\left ( \frac{a}{p} \right )$ vaut $0$ si $a$ pair et $1$ sinon.
		\end{Def}
		\end{adjustwidth}\vspace{0.5em}
		
		Pour pouvoir exploiter le \textit{critère d'Euler} dans l'algorithme du test de primalité, on doit pouvoir calculer le \textit{symbole de Legendre} pour tout entier $n$ dont on veut tester la primalité. On introduit donc le \textit{\textbf{symbole de Jacobi}} qui est une généralisation du \textit{symbole de Legendre}, définit de la manière suivante :
		\vspace{-1.5em}\begin{adjustwidth}{1.5cm}{1.5cm} 
		\begin{Def}[Symbole de Jacobi]
			\label{Jacobi}
			Le symbole de Jacobi $\left ( \frac{a}{n} \right )$ est défini, $\forall a \in \mathbb{Z}$ et $n \in \mathbb{N}$ \underline{impair}, comme produit de symboles de Legendre, en faisant intervenir la décomposition en facteurs premiers de $n$. Si $n = p_{1}*p_{2}*...*p_{k}$ pour $k \in \mathbb{N}$ telle que $p_{1}, p_{2}, ...,p_{k}$ sont des nombres premiers non nécessairement distincts, alors :
			\[\left ( \frac{a}{n} \right ) = \left ( \frac{a}{\prod_{i \in \{1,...,k\}} p_{i}} \right ) = \prod_{i \in \{1,...,k\}} \left ( \frac{a}{p_{i}} \right )\]
		\end{Def}
		\end{adjustwidth}\vspace{0.5em}
		
		Compte tenu des notions décrites ci-dessus, on peut maintenant construire l'algorithme de test de primalité de Solovay-Strassen :\\
		
		\begin{algorithm}[H]
			\caption{Test de Solovay-Strassen}\label{TSS}
			\Donnees{un entier $n$ \underline{impair} et le nombre de répétitions $k$}
			\Pour{$i$ = $1$ jusqu'à $k$}{
				Choisir aléatoirement $a$ tel que $2 < a < n$\;
				$x \gets \left ( \frac{a}{n} \right )$\;
				\Si {$x = 0$ ou $x \not\equiv a^{\frac{n-1}{2}} \pmod n$}{
					{\Retour composé\;}
				}
			}
		\Retour probablement premier\;
		\end{algorithm}
		
		\paragraph{} Cet algorithme exploite essentiellement le \textit{critère d'Euler} (théorème \ref{CritereEuler}). En effet, pour un entier $n$ dont on veut tester la primalité et un entier $a$ quelconque telle que $2 < a < n$ :
		\begin{itemize}
		\item Si $\mathbf{a^{\frac{n-1}{2}} \not\equiv \left ( \frac{a}{n} \right ) \pmod n}$, alors $n$ est surement composé.\\
		Parmi les entiers $a$ qui ne vérifient pas le \textit{critère d'Euler} (théorème \ref{CritereEuler}), il y a évidement ceux qui ne sont pas premiers avec $n$. Si l'on trouve un tel entier $a$ (qu'il soit premier ou non avec $n$), on dit que $a$ est un \textit{\textbf{témoin de non primalité}} de $n$ (\textit{témoin d'Euler}).
			
			\vspace{-1.5em}\begin{adjustwidth}{1.5cm}{1.5cm} 
			\begin{Def}[Témoin d'Euler]
			\label{TemEuler}
				Soit un entier $n > 2$. On appelle témoin d'Euler pour $n$, tout entier $a$, telle que
				\[2 < a < n  \quad \text{et} \quad a^{\frac{n-1}{2}} \not\equiv \left ( \frac{a}{n} \right ) \pmod n\]
			\end{Def}
			\end{adjustwidth}\vspace{0.5em}
		
		\item Si $\mathbf{a^{\frac{n-1}{2}} \equiv \left ( \frac{a}{n} \right ) \pmod n}$, on ne peut pas conclure avec certitude que $n$ est premier puisque la réciproque du \textit{critère d'Euler} (théorème \ref{CritereEuler}) est fausse.\\
		Un nombre $n$ vérifiant cette équation peut être premier, mais aussi composé, dans ce cas $n$ est dit \textit{\textbf{pseudo-premier d'Euler-Jacobi} de base $a$} ou menteur.
					
			\vspace{-1.5em}\begin{adjustwidth}{1.5cm}{1.5cm} 
			\begin{Def}[Nombre pseudo-premier d'Euler-Jacobi]
				\label{PseudoPremEulerJ}
				Un nombre pseudo-premier d'Euler-Jacobi de base $a$ est un nombre composé impair $n$ premier avec $a$ et tel que la congruence suivante soit vérifiée :
				\[a^{\frac{n-1}{2}} \equiv \left ( \frac{a}{n} \right ) \pmod n\]
			\end{Def}
			\end{adjustwidth}\vspace{0.5em}
			
		À la différence du test de primalité de Fermat, pour chaque entier composé $n$, au moins la moitié de tous les $a$ sont des témoins d’Euler. Par conséquent, il n’y a aucune valeur de $n$ pour laquelle tous les $a$ sont des menteurs, alors que c'est le cas pour les nombres de \textit{Carmichael} dans le test de Fermat.
		\end{itemize}
	
	
	\subsubsection{Complexité}
		\paragraph{}Pour étudier la complexité du test Solovay-Strassen, il faut étudier l'évaluation du symbole de Jacobi. En effet, dans le corps de la boucle, à la différence du test de Fermat, avant d'effectuer l'exponentiation modulaire nous allons évaluer un symbole de Jacobi. La complexité d'une itération sera de l'ordre du terme dominant entre le symbole de Jacobi et l'exponentiation modulaire.
		\paragraph{}Le symbole de Jacobi $\left ( \frac{a}{b} \right )$ s'évalue en $O(log(a) \cdot log(b))$. Ainsi, dans le cadre de notre test, le symbole de Jacobi s'évalue en $O(log(n)^{2})$.
		\paragraph{}L'évaluation du symbole de Jacobi reste dominée par l'exponentiation rapide même couplée d'une multiplication modulaire rapide.\\
				On rappelle de complexités en $O(log(n)^{2} \cdot log(log(n)) \cdot log(log(log(n))))$ en utilisant la multiplication FFT et en $ O(log(n)^{1 + log_{2}(3)})$ (avec $1 + log_{2}(3) > 2$) avec la muliplication de Karatsuba.
				
	\subsubsection{Preuve}
		\paragraph{}Durant cette démonstration, nous allons utiliser et admettre le théorème de Lagrange.
		\vspace{-1.5em}\begin{adjustwidth}{1.5cm}{1.5cm} 
		\begin{Th}[Théorème de Lagrange]
			\label{ThLagrange}
			Soient $p$ un nombre premier et $f(X) \in \mathbb{Z}[X]$ un polynôme à coefficients entiers alors:
			\begin{itemize}
			\item Soit tous les coefficients de $f$ sont divisibles par $p$.
			\item Soit $f(X) \equiv 0 \text{ mod } p$ admet au plus $degre(f)$ solutions non équivalentes.
			\end{itemize}
		\end{Th}
		\end{adjustwidth}\vspace{0.5em}
		
		\paragraph{}En admettant le théorème de Lagrange, il suffit de prouver le critère d'Euler pour avoir une preuve satisfaisante du théorème. En effet, comme énoncé précédemment, le test de Solovay-Strassen s'appuie directement sur la contraposée du critère d'Euler.
		\paragraph{}Nous partirons de plusieurs constatations pour prouver le critère d'Euler :\\
		\begin{enumerate}
		\item D'après le théorème de Lagrange, comme $p$ est premier, $x^{2} \equiv a \text{ mod } p$ admet au plus deux solutions distinctes pour chaque $a$ différent. Donc, hormis $x = 0$, comme chaque racine $x$ peut être accompagnée d'une deuxième racine comme solution de l'équation, il y a au moins $(p - 1)/2$ résidus quadratiques $a$ différents. 
		\item $(\mathbb{Z}/p\mathbb{Z}, +, \cdot)$ est un corps ($p$ premier).
		\end{enumerate}
		
		\paragraph{}Pour commencer on va partir du théorème de Fermat et le réécrire.
		\[a^{p-1} \equiv 1 \text{ mod } p \iff (a^{\frac{p-1}{2}} - 1) \cdot (a^{\frac{p-1}{2}} + 1) \equiv 0 \text{ mod } p\]
		Grâce à la remarque (2), on obtient que le produit est nul si et seulement si l'un au moins des facteurs est nul.
		
		\paragraph{}Si $a$ est un résidu quadratique, il existe $x$ tel que $x^{2} \equiv a \text{ mod } p$, on a :\\
		\[
			a^{\frac{p-1}{2}} \equiv (x^{2})^{\frac{p-1}{2}} \equiv x^{p-1} \equiv 1 \text{ mod } p
		\]
		La dernière étape est obtenue à l'aide du petit théorème de Fermat.
		
		\paragraph{}On sait que d'après le théorème de Lagrange, $(a^{\frac{p-1}{2}} - 1) \equiv 0 \text{ mod } p$ admet au plus $(p-1)/2$ solutions pour $a$. On sait également (constatation 1) qu'il y a au moins $(p-1)/2$ résidus quadratiques modulo $p$.
		\paragraph{}Donc, il y a exactement $(p-1)/2$ valeurs qui annulent le premier facteur : les résidus quadratiques. Et, les autres $(p-1)/2$ valeurs non-résidues annulent forcément le second terme pour satisfaire le petit théorème de Fermat.
		\paragraph{}En résumé :
			\begin{itemize}
				\item Si $a$ est un résidu quadratique modulo $p$, alors $a^{\frac{n-1}{2}} \equiv 1 \pmod p$.
				\item Si $a$ n'est pas un résidu quadratique modulo $p$, alors $a^{\frac{n-1}{2}} \equiv -1 \pmod p$.
			\end{itemize}
