\subsection{Test de Fermat}
	Le test de Fermat est un test de primalité probabiliste basé sur le \textit{petit théorème de Fermat} :
	
	\vspace{-1.5em}\begin{adjustwidth}{1.5cm}{1.5cm} 
	\begin{Th}[Petit théorème de Fermat (énoncé 1)]
		\label{ThFermat1}
		Si $p$ est un nombre premier, alors pour tout nombre entier $a$ premier avec $p$
		\[a^{p-1}\equiv 1 \pmod p\]
	\end{Th}
	\end{adjustwidth}\vspace{0.5em}
	
	Il existe un énoncé équivalent de ce théorème, qui est le suivant :
	
	\vspace{-1.5em}\begin{adjustwidth}{1.5cm}{1.5cm} 
	\begin{Th}[Petit théorème de Fermat (énoncé 2)]
		\label{ThFermat2}
		Si $p$ est un nombre premier, et $a$ un nombre entier quelconque, alors
		\[a^{p}\equiv a \pmod p\]
	\end{Th}
	\end{adjustwidth}\vspace{0.5em}
	
	Ce théorème doit son nom à \textit{Pierre de Fermat}, qui l'énonce la première fois en 1640. 
	
	\subsubsection{Algorithme}
		Le premier énoncé du \textit{théorème de Fermat} va être exploité pour construire l'algorithme de test de primalité. Ce théorème décrit une propriété commune à tous les nombres premiers qui peut être utilisée pour détecter si un nombre est premier ou bien composé.\\
		\indent En effet, pour un entier $n$ dont on veut tester la primalité et un entier $a$ quelconque tel que $1 < a < n$ : 
		\begin{itemize}
			\item Le fait de choisir $1 < a < n$ garantit que si $n$ était premier, $a$ sera forcément premier avec $n$ (puisque $a < n$) et ainsi le test n'échouera pas. 
			\item Si $\mathbf{a^{n-1} \not\equiv 1 \pmod n}$, alors $n$ est surement composé.\\
			Parmi les entiers $a$ qui ne vérifient pas l'inégalité de Fermat, il y a évidement ceux qui ne sont pas premiers avec $n$. Si l'on trouve un tel entier $a$ (qu'il soit premier ou non avec $n$), on dit que $a$ est un \textit{\textbf{témoin de non primalité}} de $n$ issu de la divisibilité (\textit{témoin de Fermat)}.
					
					\vspace{-1.5em}\begin{adjustwidth}{1.5cm}{1.5cm} 
					\begin{Def}[Témoin de Fermat]
						\label{TemFermat}
						Soit un entier $n \geqslant 2$. On appelle témoin de Fermat pour $n$, tout entier $a$, tel que
						\[1 < a < n  \quad \text{et} \quad a^{n-1} \not\equiv 1 \pmod n\]
					\end{Def}
					\end{adjustwidth}\vspace{0.5em}
					
			\item Si $\mathbf{a^{n-1}\equiv 1 \pmod n}$, on ne peut pas conclure avec certitude que $n$ est premier puisque la réciproque du \textit{théorème de Fermat est fausse} (théorème \ref{ThFermat1}).\\
				Un nombre $n$ vérifiant cette équation peut être premier, mais aussi composé, dans ce cas $n$ est dit \textit{\textbf{pseudo-premier} de base $a$} ou menteur.
					
					\vspace{-1.5em}\begin{adjustwidth}{1.5cm}{1.5cm} 
					\begin{Def}[Nombre pseudo-premier]
						\label{PseudoPrem}
						Un nombre pseudo-premier est un nombre premier probable (un entier naturel qui partage une propriété commune à tous les nombres premiers) qui n'est en fait pas premier. Un nombre pseudo-premier provenant du théorème de Fermat est appelé nombre pseudo-premier de Fermat.
					\end{Def}
					\end{adjustwidth}\vspace{0.5em}
					
				Si un nombre pseudo-premier $n$ de base $a$ est pseudo-premier pour toutes les valeurs de $a$ qui sont premières avec $n$ est appelé \textit{\textbf{nombre de Carmichael}}. 
			
					\vspace{-1.5em}\begin{adjustwidth}{1.5cm}{1.5cm} 
					\begin{Def}[Nombre de Carmichael]
						\label{Carmich}
						Un entier positif composé $n$ est appelé nombre de Carmichael si pour tout entier $a$ premier avec $n$,
						\[a^{n-1}\equiv 1 \pmod n\]
					\end{Def}
					\end{adjustwidth}\vspace{0.5em}
					
				L'entier $n = 561 = 3\ .\ 11\ .\ 17$ est le plus petit nombre de Carmichael puisque $a^{560} \equiv 1 \pmod 561$ pour tout entier $a$ premier avec $561$. Les nombres de Carmichael sont très rares. Il existe par exemple seulement $246\ 683$ nombres de Carmichael inférieurs à $10^{16}$. Le nombre de premiers inférieurs à $10^{16}$ est quant à lui égal à $279\ 238\ 341\ 033\ 925$. Donc la probabilité qu'un nombre premier inférieur à $10^{16}$ soit un nombre de Carmichael est plus petite que $1/10^{9}$.
			
		\end{itemize}
		
		\paragraph{}Les nombres pseudo-premiers et les nombre de Carmichael sont relativement rares. On peut donc envisager d'adopter ce critère pour un test probabiliste de primalité, qui est le \textit{test de Fermat}. En effet, le test va être répété $k$ fois, et à chaque itération, on effectue le test avec une base $a$ différente. Plus le nombre de répétitions est grand, plus la probabilité que le résultat du test soit correct augmente.
		
		\paragraph{} L'algorithme de test de primalité qu'on obtient finalement est le suivant :\\
		
		\begin{algorithm}[H]
			\caption{Test de Fermat}\label{TF}
			\Donnees{un entier $n$ et le nombre de répétitions $k$}
			\Pour{$i$ = $1$ jusqu'à $k$}{
				Choisir aléatoirement $a$ tel que $1 < a < n$\;
				\Si {$a^{n-1} \not\equiv 1 \pmod n$}
					{\Retour composé\;}
			}
		\Retour probablement premier\;
		\end{algorithm}
		
		
	\subsubsection{Complexité}
		\paragraph{}La complexité de ce test va dépendre de l'exponentiation modulaire utilisée dans le corps de la boucle de $k$ itérations.\\
		
		\begin{algorithm}[H]
			\caption{Square-and-Multiply (Left-to-right binary method)}\label{SqM}
			\Donnees{$c$, $d$, et $n$ entiers : avec $d = \sum_{i=0}^{k-1} d_{i} \cdot 2^{i}$ et $d_{k-1} = 1$}
			\Sortie{$c^d$ modulo $n$}
			$x \gets c$\;
			\Pour{$i$ = $k-2$ jusqu'à $0$}{
				$x \gets x^{2}$ mod $n$\;
				\Si {$d_{i} = 1$}
					{$x \gets x \cdot c$ mod $n$\;}
			}
		\Retour $x$\;
		\end{algorithm}
		
		\paragraph{}L'analyse de la complexité cet algorithme (dont on va admettre la preuve ici) nous donne un temps en $log(d)$ multiplications modulaires dépendantes de $n$.\\
		\indent Appliqué au contexte du \textit{test de Fermat}, on a un exposant $n-1$ et des multiplications modulaires dépendantes de $n$.\\
		\indent Pour conclure, il nous reste à voir la complexité d'une multiplication modulaire. Étant donné que nous travaillons sur des entiers de grande taille, la multiplication ne sera pas en temps constant. Il existe un algorithme "naïf" et deux multiplications dites rapides que nous allons simplement citer :
			\begin{itemize}
			\item
			Une implémentation "naïve" consiste à effectuer les calculs comme pour une multiplication de primaire. À l'aide de deux boucles imbriquées, on multiplie chiffre par chiffre nos nombres.\\
			En appelant $n$ notre nombre, sa taille est de l'ordre de $log(n)$, la complexité de cette multiplication est donc $log(n)^{2}$.
			\item
			La méthode FFT (\textit{Fast Fourrier Transformation}) effectue la multiplication en complexité :
			$$log(n) \cdot log(log(n)) \cdot log(log(log(n)))$$
			\item
			La méthode \textit{Karatsuba} quant à elle calcule le résultat en complexité $log(n)^{log_{2}(3)}$.
			\end{itemize}~\\
		On obtient une complexité finale pour le \textit{test de Fermat} de :
		\[O(k \cdot log_{2}(n) \cdot C_{mult}(n))\]
		où $k$ est le nombre de répétitions dans le \textit{test de Fermat}, $n$ l'entier testé et $C_{mult}(n)$ la complexité de la multiplication modulaire.
	
	
	\subsubsection{Preuve}
		\paragraph{}Pour démontrer la correction du test de primalité, nous allons d'abord prouver le \textit{petit théorème de Fermat}. Pour cela nous nous basons sur le fait que c'est un cas particulier du \textit{théorème d'Euler} :
		
		\vspace{-1.5em}\begin{adjustwidth}{1.5cm}{1.5cm} 
		\begin{Th}[Théorème d'Euler]
			\label{ThEuler}
			Soit $n$ un naturel supérieur ou égal à 2, et $a$ un entier premier avec $n$, alors
			\[a^{\phi(n)}\equiv 1 \pmod n\]
			où $\phi$ est la fonction indicatrice d'Euler :
			\begin{align*}
				\phi \colon \mathbb{N} &\to \mathbb{N}\\
				n &\mapsto | \{k \text{, } 1 \leq k \leq n \text{ et } pgcd(k,n) = 1\} |
			\end{align*}
		\end{Th}
		\end{adjustwidth}\vspace{0.5em}
		
		La fonction $\phi$ évaluée sur un nombre premier $p$ vaut $p - 1$. Le \textit{petit théorème de Fermat} est donc une application du théorème d'Euler en remplaçant $n$ par un nombre premier $p$.
		
		\paragraph{Preuve du \textit{théorème d'Euler} :} 
		\paragraph{}En prouvant le \textit{théorème d'Euler}, on aura prouvé \textit{petit le théorème de Fermat}, on conclura ensuite avec le preuve de l'algorithme.
			
		\paragraph{}On effectuera les calculs dans le groupe multiplicatif $(\mathbb{Z}/n\mathbb{Z})^*$, l'ensemble des naturels inférieurs à $n$ inversibles modulo $n$, ou de manière équivalente, l'ensemble des naturels inférieurs à $n$ premiers avec $n$.
		
		\paragraph{}On considère l'application suivante, avec $\alpha \in (\mathbb{Z}/n\mathbb{Z})^*$:
			\begin{align*}
				\Gamma_{\alpha} \colon (\mathbb{Z}/n\mathbb{Z})^* &\to (\mathbb{Z}/n\mathbb{Z})^*\\
				x &\mapsto \alpha \cdot x
			\end{align*}
			C'est une bijection. En effet son application inverse est $\Gamma_{\alpha^{-1}}$. $\alpha \in (\mathbb{Z}/n\mathbb{Z})^*$, donc il existe $\alpha^{-1} \in (\mathbb{Z}/n\mathbb{Z})^*$ inverse de $\alpha$ modulo $n$. C'est également une permutation (bijection d'un ensemble vers lui-même), on a donc :
			\begin{align*}
				\prod_{x \in (\mathbb{Z}/n\mathbb{Z})^*} x &= \prod_{x \in (\mathbb{Z}/n\mathbb{Z})^*} \Gamma_{\alpha}(x)\\
				&= \alpha^{|(\mathbb{Z}/n\mathbb{Z})^* |} \cdot \prod_{x \in (\mathbb{Z}/n\mathbb{Z})^*} x\\
				&= \alpha^{\phi(n)} \cdot \prod_{x \in (\mathbb{Z}/n\mathbb{Z})^*} x
			\end{align*}
			$\prod_{x \in (\mathbb{Z}/n\mathbb{Z})^*} x$ est inversible (produit d'éléments inversibles), donc on simplifie :
			\[\alpha^{\phi(n)} \equiv 1 \text{ modulo } n \]
		
		\paragraph{Preuve de l'algorithme :} directement, en se basant sur la contraposée du \textit{petit théorème de Fermat} on a :
			\begin{center}
			Soit $a$ un entier premier avec $p$, alors $a^{p-1} \not\equiv 1 \text{ modulo } n \Rightarrow p$ non premier.
			\end{center}
		
		\paragraph{Remarque :} on voit bien avec cette preuve qu'on peut affirmer avec certitude qu'un nombre qui ne passe pas le test est composé. Cela dit, si le test passe, ceci ne confirme pas forcément que notre nombre est premier.
		
