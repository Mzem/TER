\section{Architecture de l'application}

	\subsection{Organigramme et données échangées}
	L'application développée dans ce projet porte essentiellement sur l'implémentation des tests de primalité, avec quelques fonctionnalités supplémentaires. La manière dont l'application est structurée va permettre de faciliter son extension et sa réutilisation.\\
	\indent Cet organigramme représente la décomposition en modules de l'application ainsi que les informations qui circulent entre ces modules.
	\begin{figure}[H]
		\begin{tikzpicture}
		\begin{scope}[xscale=2,yscale=0.9]
			
			\node (Fct) at (0,5) [rectangle,draw,text depth=2cm,minimum width=12cm,minimum height=2.5cm,font=\textbf\Large] {\begin{tabular}{c}Fonctionnalités\end{tabular}};
			\node (CS) [rectangle,draw,dashed] at ([xshift=-2cm]Fct.center) {\begin{tabular}{c}Cryptosystèmes\end{tabular}};
			\node (TPrim) [rectangle,draw,dashed] at ([xshift=2cm]Fct.center) {\begin{tabular}{c}Tests de primalité\end{tabular}};
		
			\node (MPerf) at (0,1) [rectangle,draw,text depth=-1.5cm,minimum width=5cm,minimum height=2.5cm,font=\textbf\Large] {\begin{tabular}{c}Mesures de performance\end{tabular}};
			\node (MTps) [rectangle,draw,dashed] at ([yshift=0.3cm]MPerf.center) {\begin{tabular}{c}Mesure temps\end{tabular}};
			
			\draw[-triangle 45,blue!60] (CS.2.5) -- node[anchor=south,yshift=0.0cm]{1} (TPrim.177.5);
			\draw[-triangle 45,blue!60] (TPrim.182.5) -- node[anchor=south,yshift=-0.5cm]{2} (CS.357.5);
			
			\draw[-triangle 45,blue!60] (MTps.60) -- node[anchor=north,yshift=0.0cm]{3} (TPrim.300);
			\draw[-triangle 45,blue!60] (TPrim.250) -- node[anchor=south,xshift=-0.5cm,yshift=-0.3cm]{4} (MTps.110);
		\end{scope}
		%Légende
		\begin{scope}
			\node (LEGENDE) at (-7,-2) {\textbf{Légende :}};
			\node (PACKAGE) at (-4.5,-2) [rectangle,draw] {\begin{tabular}{c}Package\end{tabular}};
			\node (MODULE) at (-2,-2) [rectangle,draw,dashed] {\begin{tabular}{c}Module\end{tabular}};
			\path[->,>=stealth',blue!60] (0.5,-2.3) edge[bend left=0] node[anchor=south,above]{informations transmises} (3,-2.3);
		\end{scope}
		\end{tikzpicture}
		\caption{Organigramme des différents modules de l'application}\label{fig:M1}
	\end{figure}
		
	\vspace{1em}
	\hspace{-1.3em}\textbf{Notes :}\\
		\textbf{(1)} Nombre entier à tester (primalité)\\
		\textbf{(2)} Réponse sur la primalité (0 composé, 1 probablement premier, 2 premier)\\
		\textbf{(3)} Nombre entier à tester (primalité)\\
		\textbf{(4)} Réponse sur la primalité (0 composé, 1 probablement premier, 2 premier)\\
		
	
	\subsection{Fonctionnalités des modules}
		\subsubsection*{Package Fonctionnalités}
			\begin{enumerate}[leftmargin=*]
				\item Module Cryptosystèmes : implémentation de cryptosystèmes ayant recours à des nombres premiers (\textbf{\textit{RSA}}) et des générateurs aléatoires de nombres premiers (RPNG) :
				\begin{itemize}
					\item Cryptosystème RSA
					\item RPNG avec test de primalité déterministe
					\item RPNG avec test de primalité probabiliste
					\item RPNG optimal
				\end{itemize}
				\item Module Tests de primalité : implémentation de différents algorithmes de test de primalité qui feront l'objet d'une étude comparative par la suite :
				\begin{itemize}
					\item Test Naïf
					\item Test de Wilson
					\item Test de Fermat
					\item Test de Miller-Rabin
					\item Test de Solovay-Strassen
					\item Test AKS
				\end{itemize}
			\end{enumerate}
			
		\subsubsection*{Package Mesures de performance}
			\begin{enumerate}[leftmargin=*]
				\item Module Mesure temps : mesure du temps d'exécution de chaque test de primalité selon le la taille en bits du nombre à tester.
			\end{enumerate}
	
	\subsection{Outils et langages de programmation}
	Notre application va être implémentée dans le langage {\ttfamily C}. Le langage {\ttfamily C} possède plusieurs types pour représenter des nombre entiers. Cependant, tous ces types ont une précision fixe et ne peuvent pas dépasser un certain nombre d'octets. Le type le plus grand est le {\ttfamily long long int} qui peut contenir des entiers d'un taille maximale de 64 bits. Or, tous ces types sont beaucoup trop courts pour les applications cryptographiques qui nécessitent la manipulation de données d'au moins 512 bits.
	\paragraph{}{\ttfamily GNU MP} pour {\ttfamily GNU Multi Precision}, souvent appelée \lstinline!GMP! est une bibliothèque {\ttfamily C}/{\ttfamily C++} de calcul multiprécision sur des nombres entiers, rationnels et à virgule flottante qui permet en particulier de manipuler de très grand nombres.
	
	\paragraph{} Finalement, le logiciel {\ttfamily Gnuplot} va être utilisé pour faire des représentations graphiques à partir des résultats issus des mesures de performance de notre application.
	