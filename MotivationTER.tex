\documentclass[10pt]{letter}
\usepackage[utf8]{inputenc}
\usepackage[T1]{fontenc}
\usepackage[french]{babel}
\usepackage{setspace,lipsum}
\usepackage[top=3cm, bottom=3cm, right=3.5cm, left=3.5cm]{geometry}
\linespread{1.5}
\pagestyle{empty}

\begin{document}
\sffamily
	
	\begin{flushleft}
		\singlespacing
		\textbf{Sonny Klotz}, \textbf{Idir Hamad},\\ \textbf{Younes BenYamna} et \textbf{Malek Zemni}\\[.35ex]
		\footnotesize
		Université de Versailles Saint-Quentin-En-Yvelines 
	\end{flushleft}
	
	\begin{flushright}
		\singlespacing
		\textbf{Alex Gélin}\\[.35ex]
	\end{flushright}
 
	\begin{flushright}Versailles, le \today\\\end{flushright}

	\vspace{1em}
	
	\textbf{Objet :} candidature pour le projet de cryptographie \guillemotleft Logarithme Discret\guillemotright
	
	Monsieur,
	
	\hspace{1cm} Notre groupe, formé par les étudiants dont les noms sont cité ci-dessus, est ... pour traiter le sujet du Logarithme Discret....
	(ici on parle du projet pourquoi on l'a choisi)
	
	\hspace{1cm} Par ailleurs, notre groupe est composé uniquement d'étudiants qui ont suivi le module de cryptographie en 3 ème année de licence informatique. De plus, certains membres du groupe ont obtenu les meilleures notes de ce modules. Ceci fait de nous un groupe bien conscient de la question dont traite le sujet du projet........
	(ici on parle de nous, nos compétences, pourquoi nous et pas d'autres)
	
	\hspace{1cm} Avec l’espoir de vous convaincre de notre enthousiasme, nous vous prions d’agréer, Monsieur, nos respectueuses salutations.

\end{document}