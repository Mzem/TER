\documentclass[10pt]{letter}
\usepackage[utf8]{inputenc}
\usepackage[T1]{fontenc}
\usepackage[french]{babel}
\usepackage{setspace,lipsum}
\usepackage[top=3cm, bottom=3cm, right=3.5cm, left=3.5cm]{geometry}
\linespread{1.5}
\pagestyle{empty}

\begin{document}
\sffamily
	
	\begin{flushleft}
		\singlespacing
		\textbf{Sonny Klotz}, \textbf{Idir Hamad},\\ \textbf{Younes BenYamna} et \textbf{Malek Zemni}\\[.35ex]
		\footnotesize
		Université de Versailles Saint-Quentin-En-Yvelines 
	\end{flushleft}
	
	\begin{flushright}
		\singlespacing
		\textbf{Alex Gélin}\\[.35ex]
	\end{flushright}
 
	\begin{flushright}Versailles, le \today\\\end{flushright}

	\vspace{1em}
	
	\textbf{Objet :} candidature pour le projet de cryptographie \guillemotleft Logarithme Discret\guillemotright
	
	Monsieur,
	
	\hspace{1cm} Notre groupe, formé par les étudiants dont les noms sont cité ci-dessus, est motivé pour traiter le sujet du Logarithme Discret.
	
	\hspace{1cm} Tout d'abord pour nous présenter, nous pouvons dire que nous sommes étudiants à l'UVSQ depuis la licence, nous avons donc suivi le module cryptographie en L3 en addition du module de cryptopgraphie pour M1. Trois des quatres membres du groupe (Idir, Malek et Younes) ont pour vocation de poursuivre avec un M2 Secrets d'où l'engouement pour un sujet portant sur la cryptographie.
	
	\hspace{1cm} Par ailleurs, en ce qui concerne notre aptitude à traiter le sujet, nous sommes familiers avec LaTeX et GMP. En effet, LaTeX est l'outil avec lequel nous avons écrit cette lettre, mais aussi, le module Cryptographie de licence nous a permis de découvrir la librairie GMP avec le développement des tests de primalité de Fermat et de Miller-Rabin en langage C.
	
	\hspace{1cm} Nous espérons avec ce sujet nous perfectionner et étendre notre culture sur des concepts clés (clé secrète lol) de la cryptographie.
	
	\hspace{1cm} Avec l’espoir de vous convaincre de notre enthousiasme, nous vous prions d’agréer, Monsieur, nos respectueuses salutations.
	
	\begin{flushright}
		Cordialement.
	\end{flushright}
\end{document}
